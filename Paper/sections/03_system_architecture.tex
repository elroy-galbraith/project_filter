\section{System Architecture}

TRIDENT implements a three-layer dispatcher-support architecture where each component provides independent value while contributing to intelligent queue prioritization. The system does not make clinical triage decisions---those remain with trained dispatchers applying ESI or START protocols---but ensures dispatchers receive the highest-priority calls first along with structured information to support protocol application. Figure~\ref{fig:system_diagram} illustrates the system flow.

\begin{figure}[ht]
\centering
\includegraphics[width=0.9\textwidth]{figures/project_filter_architecture.pdf}
\caption{The TRIDENT architecture. The system processes raw audio through two parallel streams: (Left) A Caribbean-adapted ASR and NLP pipeline for entity extraction and content analysis, and (Right) a bio-acoustic analysis layer for detecting physiological distress markers. The \textbf{Queue Prioritization Engine} integrates three independent signals---transcription confidence, extracted clinical indicators, and vocal distress---to determine queue position for dispatcher attention. This ensures that (1) calls with low transcription confidence but high vocal distress receive immediate human review, and (2) semantically urgent calls from calm reporters are not delayed due to absent vocal stress markers. The dispatcher then applies established triage protocols (ESI for routine operations, START for mass casualty events) using both TRIDENT's extracted entities and direct audio review.}
\label{fig:system_diagram}
\end{figure}

\subsection{Design Philosophy: Enabling Protocol Application}

TRIDENT's architecture reflects a core principle: \textbf{AI should empower dispatchers to apply established protocols more effectively, not replace clinical judgment}. Caribbean health ministries have adopted validated triage frameworks, ESI for emergency departments, START for mass casualty incidents, that represent decades of clinical refinement. TRIDENT's role is to solve the \emph{input problem}: ensuring these protocols can be applied equitably to Caribbean-accented callers whose speech current ASR systems fail to transcribe accurately.

Each architectural layer addresses a specific input challenge:

\begin{itemize}
    \item \textbf{Layer 1 (ASR):} Produces transcripts and confidence scores, enabling dispatchers to know when to trust text versus listen directly to audio.
    
    \item \textbf{Layer 2 (NLP):} Extracts structured clinical entities---location, mechanism of injury, breathing status, vulnerable populations---that map directly to ESI/START decision points.
    
    \item \textbf{Layer 3 (Bio-Acoustic):} Detects physiological distress markers that indicate caller crisis state, providing a signal not currently captured by standard protocols but valuable for queue prioritization.
\end{itemize}

The following subsections detail each layer's implementation.

\subsection{Layer 1: Caribbean-Tuned ASR}

The ASR layer employs OpenAI's Whisper Large model (769M parameters) fine-tuned with Low-Rank Adaptation (LoRA) on Caribbean broadcast speech. Competition experience suggests that Whisper Large is more accurate than Whisper Medium for Caribbean speech, even though the latter is the default model used by OpenAI. This notwithstanding, for TRIDENT, we selected Whisper Medium over Large based on Madden et al.'s \cite{madden2025} scaling law, which demonstrates diminishing returns from model size compared to domain-specific data for Caribbean varieties. Furthermore, Whisper Medium is more efficient to run on a Raspberry Pi 5, which is the edge device we are using for TRIDENT.

\textbf{Fine-tuning Configuration:}
\begin{itemize}
    \item Base model: openai/whisper-medium
    \item Adaptation: LoRA (rank=16, alpha=32)
    \item Training data: BBC Caribbean broadcast corpus ($\sim$28,000 clips)
    \item Trainable parameters: $\sim$0.5\% of total model
\end{itemize}

\textbf{Confidence Scoring:} The system computes \textbf{utterance-level} confidence as the mean log-probability across all decoded tokens, normalized to a 0-1 scale. Specifically:

\begin{equation}
\text{confidence} = \exp\left(\frac{1}{N}\sum_{i=1}^{N} \log P(t_i | t_1 \ldots t_{i-1}, \text{audio})\right)
\end{equation}

We use utterance-level rather than token-level confidence because emergency triage requires a holistic assessment of transcription reliability. Token-level confidence would require additional aggregation logic and may miss systematic degradation patterns (e.g., consistently low confidence across an entire basilectal utterance).

\textbf{Confidence Threshold:} We set the ``low confidence'' threshold at 0.7 based on initial calibration experiments, though sensitivity analysis is needed to optimize this value (see Limitations).

The configuration above reflects design specifications informed by the Caribbean Voices AI Hackathon, which provided access to the BBC Caribbean corpus. Competition data rights are retained by the organizers; empirical validation therefore remains future work. The architecture accommodates any Caribbean-tuned Whisper model meeting these specifications.

\subsection{Layer 2: Local NLP Entity Extraction}

When ASR produces usable transcription (confidence $\geq$ 0.7), the NLP layer extracts structured emergency information using Llama 3 8B running locally via Ollama. The extraction schema targets entity types that map directly to ESI and START triage protocol decision points.

\subsubsection{Entity Extraction Schema}

The schema targets four entity categories:
\begin{itemize}
    \item \textbf{LOCATION:} Street addresses, landmarks, geographic references
    \item \textbf{MECHANISM/HAZARD:} Emergency type (fire, flood, medical, violence, traffic)
    \item \textbf{CLINICAL INDICATORS:} Breathing status, consciousness, bleeding, mobility
    \item \textbf{SCALE:} Number of people involved, vulnerable populations
\end{itemize}

\subsubsection{Mapping to Triage Protocols}

TRIDENT entities support ESI and START protocol application. For ESI, extracted entities inform the four decision points: Point A (lifesaving intervention) captures "not breathing," "choking," "unresponsive"; Point B (high-risk situation) captures mechanism of injury and altered status; Point C (resource needs) uses hazard type and complexity; Point D (vital signs) uses reported vitals and distress indicators \cite{esi_handbook}.

For mass casualty events using START, entities support rapid sorting: GREEN captures "walking," "minor injuries"; YELLOW captures "injured but stable," "conscious"; RED captures "trapped," "not breathing," "heavy bleeding"; BLACK captures cessation indicators.

\begin{table}[ht]
\centering
\small
\begin{tabular}{@{}p{2.5cm}p{4cm}p{5.5cm}@{}}
\toprule
\textbf{Protocol} & \textbf{Decision Point} & \textbf{Example Extraction Target} \\ \midrule
ESI Level 1 & Immediate lifesaving intervention? & ``not breathing,'' ``choking,'' ``heavy bleeding,'' ``unresponsive'' \\[0.5em]
START RED & Not walking, breathing issues & ``trapped,'' ``not breathing,'' ``unresponsive,'' ``heavy bleeding'' \\ \bottomrule
\end{tabular}
\caption{Example entity extraction targets supporting ESI and START protocols. Full protocol mappings detailed in extended version.}
\label{tab:protocol_mapping_simplified}
\end{table}

\subsubsection{Handling Garbled Input}

The NLP layer handles low-quality transcriptions through confidence-aware prompting. When ASR confidence is below 0.7, the system instructs the LLM to mark uncertain extractions, avoid hallucination, prioritize location extraction, and note phonetically similar alternatives. When confidence is very low ($<$0.4), minimal structured output is produced and the call is flagged for immediate human review.

\subsubsection{Content Indicator Scoring}

The NLP layer computes a \textbf{Content Indicator Score} ($S_c \in [0,100]$) quantifying urgency implied by semantic content, independent of how the caller sounds. This addresses a critical gap: a trained first responder may report a mass casualty event calmly, producing low bio-acoustic distress despite extremely urgent content. Without content analysis, such calls would be deprioritized.

Rather than keyword matching, we leverage the LLM's semantic understanding to classify transcript content. This approach handles Caribbean creole variants (``mi granmodda drop dung an she nah move'' conveys the same urgency as ``my grandmother collapsed and she's not moving''), negation, and indirect references.

The LLM outputs structured classifications:
\begin{verbatim}
{
  "hazard_category": "violent_crime" | "medical" | "fire" |
                     "flood" | "traffic" | "infrastructure" | "other",
  "life_threat_level": "imminent" | "potential" | "none",
  "vulnerable_population": true | false,
  "situation_status": "escalating" | "stable" | "resolved",
  "persons_affected": <integer>
}
\end{verbatim}

A deterministic function maps classifications to the score:
\begin{equation}
S_c = \min\left(100,\ S_{\text{hazard}} + S_{\text{threat}} + S_{\text{vuln}} + S_{\text{scale}}\right)
\label{eq:content_severity}
\end{equation}

\textbf{Scoring components:} Hazard category weights range from 30 (violent crime) to 5 (other). Life-threat level contributes +30 (imminent), +15 (potential), or +0 (none). Vulnerable population adds +15. Scale combines persons affected (+5 per person, capped at +20) and escalation status (+10 if escalating).

\textbf{Example calculations:}

\begin{table}[ht]
\centering
\small
\begin{tabular}{@{}p{0.40\textwidth}p{0.35\textwidth}c@{}}
\toprule
\textbf{Transcript} & \textbf{Classification} & \textbf{$S_c$} \\ \midrule
``Pothole on Nelson Street'' &
infrastructure, none, false, stable, 0 & 10 \\[0.5em]
``House fire, spreading to neighbor's yard'' &
fire, potential, false, escalating, 0 & 50 \\[0.5em]
``Pickney dem trap inna di fire'' &
fire, imminent, true, stable, 2+ & 80 \\ \bottomrule
\end{tabular}
\caption{Content indicator scoring via LLM classification. Semantic understanding captures urgency from Caribbean creole variants. High scores elevate queue priority; clinical triage remains with dispatchers.}
\label{tab:severity_examples}
\end{table}

The Content Indicator Score feeds into queue prioritization (Section~\ref{sec:queue_prioritization}), ensuring semantically urgent calls reach dispatchers promptly even when vocal distress markers are absent. Weights are tunable parameters that should be calibrated with local emergency services to reflect institutional priorities and regional hazard profiles.

\subsection{Layer 3: Bio-Acoustic Distress Detection}

The bio-acoustic layer operates on raw audio, independent of ASR success, extracting features correlated with psychological distress. Based on the vocal stress literature \cite{schmalz2025, vanpuyvelde2018, veiga2025}, we focus on features that capture physiological arousal through vocal production changes.

\subsubsection{Feature Extraction}

Using librosa, we extract the following acoustic features:

\begin{enumerate}
    \item \textbf{Fundamental Frequency (F0):} Mean pitch extracted via autocorrelation method
    \begin{itemize}
        \item Typical baseline: 85--180 Hz (male), 165--255 Hz (female) \cite{titze1989}
        \item Stress indicator: Elevation above speaker baseline
    \end{itemize}

    \item \textbf{F0 Coefficient of Variation (CV):} Pitch instability measure
    \begin{itemize}
        \item Computed as $CV = \sigma_{F0} / \mu_{F0}$
        \item Normalizes for baseline differences across speakers
        \item Stress indicator: $CV > 0.3$ suggests vocal instability
    \end{itemize}

    \item \textbf{Energy (RMS amplitude):} Mean intensity across utterance
    \begin{itemize}
        \item Normalized to 0--1 scale relative to recording gain
        \item Stress indicator: Elevated intensity during distress vocalizations
    \end{itemize}

    \item \textbf{Jitter:} Cycle-to-cycle variation in F0 period
    \begin{itemize}
        \item Relatively independent of prosodic patterns \cite{vanpuyvelde2018}
        \item Pathology threshold: $>$1.04\% \cite{boersma2013}
    \end{itemize}
\end{enumerate}

\subsubsection{Distress Score Calculation}

The distress score combines multiple acoustic indicators into a composite metric. We weight features according to their documented reliability and sex-independence:

\begin{align}
D &= w_{\text{pitch}} \cdot P + w_{\text{var}} \cdot V + w_{\text{energy}} \cdot E + w_{\text{jitter}} \cdot J
\label{eq:distress}
\end{align}

\noindent where:

\begin{align}
P &= \min\left(1.0, \max\left(0, \frac{\bar{F_0} - 180}{120}\right)\right) & \text{(pitch elevation)} \\
V &= \min\left(1.0, \frac{CV_{F0}}{0.5}\right) & \text{(pitch instability)} \\
E &= \min\left(1.0, \frac{\bar{E}}{0.1}\right) & \text{(energy)} \\
J &= \min\left(1.0, \frac{\text{jitter}}{0.02}\right) & \text{(perturbation)}
\end{align}

The weights reflect relative reliability from the literature:
\begin{itemize}
    \item $w_{\text{pitch}} = 0.30$ --- F0 elevation is the most consistent stress marker but is sex-dependent
    \item $w_{\text{var}} = 0.35$ --- F0 coefficient of variation is sex-normalized and robust
    \item $w_{\text{energy}} = 0.20$ --- intensity elevation accompanies distress
    \item $w_{\text{jitter}} = 0.15$ --- perturbation measures are prosody-independent
\end{itemize}

\subsubsection{Threshold Classification}

\begin{itemize}
    \item \textbf{High Distress:} $D > 0.5$
    \item \textbf{Low Distress:} $D \leq 0.5$
\end{itemize}

These thresholds are calibrated against Van Puyvelde et al.'s \cite{vanpuyvelde2018} findings on vocal markers in emergency versus baseline speech.

\textbf{Note on sex differences:} The distress score prioritizes sex-normalized features (CV, jitter) over absolute F0 elevation to mitigate the substantial baseline differences between male (85--175 Hz) and female (165--270 Hz) speakers. See Section~\ref{sec:sex_limitations} for detailed discussion of remaining bias risks.
\subsection{The Complementarity Principle}

The theoretical foundation for our multi-layer design rests on what we term the \textbf{Complementarity Principle}: the three signal dimensions capture distinct failure modes and urgency indicators that compensate for each other's blind spots, ensuring dispatchers receive the most critical calls first regardless of which individual signal might fail.

\textbf{Dimension 1: Transcription Confidence.} The conditions that degrade ASR performance (high stress, code-switching to basilect, environmental noise) are precisely the conditions that often accompany genuine emergencies. Low confidence is not merely a technical limitation to be hidden---it correlates with caller distress and should elevate queue priority while flagging the call for direct audio review.

\textbf{Dimension 2: Content Indicators.} Semantic analysis of transcript content captures urgency that vocal characteristics may miss. Trained professionals, repeat callers, and composed bystanders often report critical emergencies without elevated vocal stress---their calm delivery masks the urgency that only content analysis reveals. When transcription confidence is high, extracted entities map directly to ESI/START decision points.

\textbf{Dimension 3: Bio-Acoustic Distress.} Vocal stress markers (elevated pitch, intensity, instability) provide a parallel assessment channel that operates on raw audio, independent of transcription success. A caller whose speech is entirely unintelligible to ASR will still produce detectable distress signals. This dimension captures information not currently used by ESI or START protocols, representing TRIDENT's novel contribution to dispatcher awareness.

This creates a robust prioritization space with complementary coverage:

\textbf{Dimensional ordering.} The three dimensions are evaluated in deliberate sequence: \emph{Confidence}, \emph{Content}, \emph{Concern}. This ordering reflects operational logic: (1) \emph{Can we understand the caller?}---ASR confidence determines whether transcription is reliable enough for downstream analysis; (2) \emph{What is being reported?}---semantic content establishes the substance of the emergency; (3) \emph{How distressed does the caller sound?}---bio-acoustic indicators validate and can elevate priority, but do not override content. This sequence ensures that a composed professional reporting a mass casualty event receives appropriate priority based on content, while a highly distressed caller reporting a minor issue is not over-prioritized based on vocal expression alone.

\begin{itemize}
    \item \textbf{High Confidence + Low Content + Low Concern:} Routine call; dispatcher applies ESI using extracted entities at normal pace
    
    \item \textbf{High Confidence + High Content + Low Concern:} The composed reporter---urgent content from a calm caller requires elevated queue position; dispatcher reviews entities and applies ESI, likely assigning ESI-2 or ESI-3
    
    \item \textbf{High Confidence + Low Content + High Concern:} Anxious caller, possibly minor issue---dispatcher assesses whether distress reflects emergency or anxiety
    
    \item \textbf{High Confidence + High Content + High Concern:} All signals aligned; immediate queue position for rapid ESI/START application
    
    \item \textbf{Low Confidence + Low Content + Low Concern:} Likely technical issue; dispatcher reviews audio quality before processing
    
    \item \textbf{Low Confidence + High Content + Low Concern:} Garbled but fragments suggest urgency---elevated priority; dispatcher listens directly
    
    \item \textbf{Low Confidence + Low Content + High Concern:} Distressed caller with unintelligible speech---immediate priority; dispatcher listens and applies protocol based on direct assessment
    
    \item \textbf{Low Confidence + High Content + High Concern:} Maximum queue priority---all indicators suggest crisis; immediate dispatcher attention
\end{itemize}

Two cells represent our key insights. The \textbf{High Confidence + High Content + Low Concern} cell captures callers whose semantic content demands urgent attention despite calm delivery: the trained first responder, medical professional, or composed bystander whose measured voice belies the severity of their report. The \textbf{Low Confidence + Low Content + High Concern} cases capture the complementary pattern---callers in crisis whose speech has shifted toward basilectal registers, where ASR failure combined with vocal stress becomes valuable prioritization information rather than system failure.

Together, these insights ensure that neither semantic nor paralinguistic signals alone determine queue position---and that clinical triage decisions remain with trained dispatchers who can assess the full context of each call.

\subsection{Queue Prioritization Engine}
\label{sec:queue_prioritization}

The Queue Prioritization Engine integrates three independent signals to determine the order in which calls receive dispatcher attention. \textbf{Critically, this system determines queue position, not clinical triage category.} Clinical triage---assigning ESI levels 1--5 or START colors (RED/YELLOW/GREEN/BLACK)---remains the responsibility of trained dispatchers applying Ministry of Health protocols.

The prioritization logic ensures that:
\begin{enumerate}
    \item Callers most likely to need immediate intervention reach dispatchers first
    \item Dispatchers receive structured information to support rapid protocol application
    \item Calls with unreliable transcriptions are flagged for direct audio review
\end{enumerate}

\subsubsection{Three-Dimensional Prioritization Space}

Each call is mapped to a point in prioritization space defined by:

\begin{itemize}
    \item \textbf{Transcription Confidence} ($C$): High ($\geq 0.7$) or Low ($< 0.7$)
    \item \textbf{Content Indicators} ($S_c$): High ($\geq 50$) or Low ($< 50$)
    \item \textbf{Bio-Acoustic Distress} ($D$): High ($> 0.5$) or Low ($\leq 0.5$)
\end{itemize}

The $2 \times 2 \times 2$ combination yields eight queue priority cells, shown in Table~\ref{tab:queue_matrix_3d}.

\begin{table}[ht]
\centering
\small
\begin{tabular}{@{}ccclp{4.8cm}@{}}
\toprule
\textbf{Confidence} & \textbf{Content} & \textbf{Concern} & \textbf{Queue} & \textbf{Dispatcher Action} \\ \midrule
High & Low & Low & \textbf{Q5-ROUTINE} & Apply ESI using extracted entities \\
High & High & Low & \textbf{Q2-ELEVATED} & Priority review; calm reporter, urgent content$^*$ \\
High & Low & High & \textbf{Q3-MONITOR} & Review for anxiety vs. emergency \\
High & High & High & \textbf{Q1-IMMEDIATE} & Immediate attention; apply ESI/START \\
Low & Low & Low & \textbf{Q5-REVIEW} & Check audio quality; possible technical issue \\
Low & High & Low & \textbf{Q2-ELEVATED} & Listen to audio; fragments suggest urgency \\
Low & Low & High & \textbf{Q1-IMMEDIATE} & Priority audio review; possible dialect shift$^\dagger$ \\
Low & High & High & \textbf{Q1-IMMEDIATE} & Highest priority; all indicators elevated \\ \bottomrule
\end{tabular}
\caption{Three-dimensional queue prioritization matrix. $^*$Addresses trained responder/composed bystander scenario. $^\dagger$Preserves core insight: low ASR confidence + high vocal concern may indicate stress-induced basilectal shift requiring human ears.}
\label{tab:queue_matrix_3d}
\end{table}

\subsubsection{Queue Priority Levels}

\begin{description}
    \item[Q1-IMMEDIATE:] Top of queue. Dispatcher reviews within seconds. System flags call for potential crisis requiring direct audio assessment.
    
    \item[Q2-ELEVATED:] High priority queue. Dispatcher attention within 1--2 minutes. Extracted entities displayed prominently to support rapid ESI/START application.
    
    \item[Q3-MONITOR:] Moderate priority. May indicate anxious caller with non-urgent situation. Dispatcher assesses and de-escalates if appropriate.
    
    \item[Q5-ROUTINE:] Standard queue. Extracted entities available; dispatcher applies ESI at normal pace.
    
    \item[Q5-REVIEW:] Standard queue but flagged for audio quality check. May indicate technical issues rather than emergency content.
\end{description}

\textbf{Note on Q4:} The current matrix does not produce a Q4 outcome. Future refinement with real operational data may identify scenarios warranting an intermediate priority level.

\subsubsection{Relationship to Clinical Triage Protocols}

Table~\ref{tab:protocol_mapping} illustrates how TRIDENT's queue prioritization relates to---but does not replace---clinical triage protocols.

\begin{table}[ht]
\centering
\small
\begin{tabular}{@{}p{2.5cm}p{5cm}p{5cm}@{}}
\toprule
\textbf{TRIDENT Output} & \textbf{Dispatcher Action} & \textbf{Protocol Application} \\ \midrule
Q1-IMMEDIATE & Immediate audio review; assess caller state & Dispatcher determines ESI-1/2 or START-RED based on clinical assessment \\[0.5em]
Q2-ELEVATED & Review extracted entities; listen if uncertain & Dispatcher applies ESI using structured data; may be ESI-2 through ESI-4 \\[0.5em]
Q3-MONITOR & Assess distress source; de-escalate if needed & Often ESI-4/5 after dispatcher determines no emergency \\[0.5em]
Q5-ROUTINE/REVIEW & Process normally using extracted metadata & Full ESI protocol application; typically ESI-3 through ESI-5 \\ \bottomrule
\end{tabular}
\caption{TRIDENT queue priority does not determine clinical triage level. Dispatchers apply ESI or START protocols after reviewing TRIDENT's structured outputs and/or call audio.}
\label{tab:protocol_mapping}
\end{table}

\subsubsection{Dispatcher Interface}

Figure~\ref{fig:ui_low_risk} and Figure~\ref{fig:ui_high_risk} illustrate the dispatcher interface for contrasting scenarios. The interface presents:

\begin{itemize}
    \item Queue priority level with visual urgency coding
    \item Transcription confidence (with recommendation to review audio if low)
    \item Extracted clinical entities mapped to ESI/START decision points
    \item Bio-acoustic distress indicators
    \item One-click access to call audio for direct assessment
\end{itemize}

\begin{figure}[ht]
\centering
\includegraphics[width=0.85\textwidth]{figures/Screenshot 2025-11-30 at 8.41.20.png}
\caption{Dispatcher interface for a routine scenario (Q5-ROUTINE). High transcription confidence enables reliable entity extraction. The dispatcher can apply ESI protocol using the structured location, hazard type, and resource need data. Audio review is available but not flagged as necessary.}
\label{fig:ui_low_risk}
\end{figure}

\begin{figure}[ht]
\centering
\includegraphics[width=0.85\textwidth]{figures/Screenshot 2025-11-30 at 8.42.43.png}
\caption{Dispatcher interface for a high-priority scenario (Q1-IMMEDIATE). Elevated distress markers combined with low transcription confidence trigger immediate queue placement. The interface prominently recommends audio review and displays partial entity extraction with uncertainty markers. The dispatcher will listen directly and apply ESI or START protocol based on their clinical assessment.}
\label{fig:ui_high_risk}
\end{figure}