\section{Introduction}

When a caller dials emergency services during a crisis, the interaction between human distress and automated systems creates a critical dependency on speech recognition accuracy. Modern ASR systems, however, exhibit well-documented performance disparities across demographic groups \cite{koenecke2020}. For Caribbean English speakers---a population of over 40 million across the Anglophone Caribbean and diaspora---these disparities compound with a linguistic phenomenon: under acute stress, speakers tend to shift toward basilectal (more creole-heavy) speech registers, precisely the varieties on which ASR systems perform worst.

\textbf{This paper presents TRIDENT (Triage via Dual-stream Emergency Natural language and Tone), an architectural framework} designed not to eliminate ASR errors on Caribbean speech---an unrealistic goal given current technology---but to build a triage system that remains functional when such errors occur. \textbf{We frame this work as a position paper and system proposal}, establishing theoretical foundations and design rationale while acknowledging that end-to-end empirical validation on Caribbean emergency calls remains future work.

\textbf{Our central and novel contribution is reframing low ASR confidence from a system failure into a triage signal that, combined with bio-acoustic distress indicators, routes calls appropriately to human dispatchers.} This insight---that ASR uncertainty $\times$ acoustic stress fusion provides a robust triage mechanism---represents a conceptual shift from treating accent-induced errors as bugs to treating them as features that correlate with genuine caller distress.

The architecture addresses four gaps in existing emergency AI systems: (1) reliance on cloud-dependent, accent-agnostic ASR; (2) exclusive focus on textual features, ignoring paralinguistic stress signals; (3) no consideration of dialect continua or stress-induced register shifting; and (4) inability to function during infrastructure failures that commonly accompany disasters.

\textbf{Note on stress-induced register shift:} While we focus on Caribbean creole continua, the phenomenon of dialect reversion under cognitive load is not unique to this population. The inhibitory control model of bilingual processing \cite{green1998} and research on the Lombard effect (speech modifications in noisy environments) suggest our framework may generalize to other bidialectal populations worldwide. Caribbean emergency services serve as our motivating case study, but the architectural principles apply broadly.
