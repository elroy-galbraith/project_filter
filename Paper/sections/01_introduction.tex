\section{Introduction}

When a caller dials emergency services during a crisis, modern automatic speech recognition (ASR) systems exhibit well-documented performance disparities across demographic groups \cite{koenecke2020}. For Caribbean English speakers---a population of over 40 million---these disparities compound with a linguistic phenomenon: under acute stress, speakers tend to shift toward basilectal (more creole-heavy) speech registers, precisely the varieties on which ASR systems perform worst.

Caribbean health ministries have adopted internationally-validated triage protocols: the Emergency Severity Index (ESI) for routine operations and START (Simple Triage and Rapid Treatment) for mass casualty events. These protocols assume dispatchers can accurately capture caller information---an assumption that fails systematically when ASR systems cannot reliably transcribe Caribbean speech.

\subsection{TRIDENT: Dispatcher-Empowered Architecture}

\textbf{This paper presents TRIDENT (\textbf{T}ranscription and \textbf{R}outing \textbf{I}ntelligence for \textbf{D}ispatcher-\textbf{E}mpowered \textbf{N}ational \textbf{T}riage)}, designed to ensure Caribbean-accented callers receive equitable access to established triage protocols. Rather than attempting to eliminate ASR errors---an unrealistic goal---we build a \textbf{dispatcher-support system} that remains functional when transcription fails.

\textbf{Our central contribution is a three-layer framework} providing dispatchers with structured inputs for protocol application:

\begin{enumerate}
    \item \textbf{Transcription confidence:} Flags unreliable transcripts so dispatchers know to listen directly to audio

    \item \textbf{Structured entity extraction:} Extracts clinical indicators (location, mechanism, breathing status, vulnerable populations) even from degraded transcriptions

    \item \textbf{Bio-acoustic distress detection:} Provides physiological stress markers independent of transcript content
\end{enumerate}

\subsection{Key Insights}
\label{sec:key_insights}

Two complementary insights motivate this design:

\begin{enumerate}
    \item \textbf{Content beyond voice:} Trained responders and composed bystanders may report life-threatening emergencies without elevated vocal stress. Semantic extraction captures information that paralinguistic features miss---ensuring ``children trapped in burning building,'' spoken calmly, provides dispatchers with structured data for appropriate triage classification.

    \item \textbf{Uncertainty as prioritization signal:} Low ASR confidence, rather than representing failure, serves as a queue prioritization indicator---particularly when combined with elevated vocal distress marking a caller in crisis whose speech may have shifted toward basilectal registers. This reframes accent-induced transcription errors from bugs into features correlating with genuine distress.
\end{enumerate}

TRIDENT addresses critical gaps in existing emergency AI---cloud dependency with accent-agnostic ASR, text-only analysis ignoring paralinguistic signals, dialect blindness to stress-induced register shifting, and infrastructure fragility during disasters---while \textbf{respecting the clinical authority of established protocols}. The system structures inputs and prioritizes queues, but triage decisions remain with trained professionals applying Ministry of Health-mandated frameworks.
