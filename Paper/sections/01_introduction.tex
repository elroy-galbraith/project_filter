\section{Introduction}

When a caller dials emergency services during a crisis, the interaction between human distress and automated systems creates a critical dependency on speech recognition accuracy. Modern ASR systems, however, exhibit well-documented performance disparities across demographic groups \cite{koenecke2020}. For Caribbean English speakers---a population of over 40 million across the Anglophone Caribbean and diaspora---these disparities compound with a linguistic phenomenon: under acute stress, speakers tend to shift toward basilectal (more creole-heavy) speech registers, precisely the varieties on which ASR systems perform worst.

\textbf{This paper presents TRIDENT (Triage via Dual-stream Emergency Natural language and Tone), an architectural framework} designed not to eliminate ASR errors on Caribbean speech---an unrealistic goal given current technology---but to build a triage system that remains functional when such errors occur. \textbf{We frame this work as a position paper and system proposal}, establishing theoretical foundations and design rationale while acknowledging that end-to-end empirical validation on Caribbean emergency calls remains future work.

\textbf{Our central contribution is a three-dimensional triage framework} that integrates ASR confidence, bio-acoustic distress, and semantic content severity. Two complementary insights motivate this design:

\begin{enumerate}
    \item \textbf{Uncertainty as signal:} Low ASR confidence, rather than representing system failure, serves as a valuable triage indicator---particularly when combined with elevated vocal distress markers indicating a caller in crisis whose speech may have shifted toward basilectal registers. This reframes accent-induced transcription errors from bugs into features that correlate with genuine caller distress.
    
    \item \textbf{Content beyond voice:} Trained first responders, medical professionals, and composed bystanders may report life-threatening emergencies without elevated vocal stress. Semantic analysis of transcript content captures urgency that paralinguistic features alone would miss---ensuring that ``children trapped in burning building,'' spoken calmly, receives appropriate priority.
\end{enumerate}

The architecture addresses four gaps in existing emergency AI systems: (1) reliance on cloud-dependent, accent-agnostic ASR; (2) exclusive focus on textual features, ignoring paralinguistic stress signals; (3) no consideration of dialect continua or stress-induced register shifting; and (4) inability to function during infrastructure failures that commonly accompany disasters.

\textbf{Note on stress-induced register shift:} While we focus on Caribbean creole continua, the phenomenon of dialect reversion under cognitive load is not unique to this population. The inhibitory control model of bilingual processing \cite{green1998} and research on the Lombard effect (speech modifications in noisy environments) suggest our framework may generalize to other bidialectal populations worldwide. Caribbean emergency services serve as our motivating case study, but the architectural principles apply broadly.