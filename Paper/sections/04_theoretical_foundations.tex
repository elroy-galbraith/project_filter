\section{Theoretical Foundations}

Fine-tuning Whisper on Caribbean speech improves transcription but cannot eliminate the accent gap. Madden et al. \cite{madden2025} achieved 30\% WER on Jamaican Patois---dramatic improvement from 89\% baseline, but still far above the $<$5\% WER typical for standard English. Moreover, fine-tuning on broadcast speech cannot capture emergency acoustics: elevated noise, emotional qualities, and stress-induced basilectal reversion. ASR alone will fail when needed most.

Conversely, bio-acoustic distress detection cannot provide semantic information needed for dispatch. A caller may exhibit extreme vocal stress while saying ``my house is on fire'' or ``I lost my keys''---identical distress signals but dramatically different responses. Furthermore, Deschamps-Berger et al. \cite{deschampsberger2021} found laboratory emotion recognition accuracy (63\%) drops substantially in real emergency calls (45.6\%). Bio-acoustic features provide gradient information about caller state but cannot substitute for semantic content.

\subsection{The Integration Thesis}

Our architecture integrates these complementary information sources based on the following thesis: \textbf{In emergency contexts, the correlation between ASR failure and genuine distress creates an opportunity to use recognition uncertainty as a routing signal rather than an error to be minimized.}

\begin{figure}[!htb]
\centering
\includegraphics[width=0.85\textwidth]{figures/trident_theoretical_model.pdf}
\caption{The TRIDENT Integration Thesis: Stress-Induced Dialect Shift vs. ASR Performance. The model illustrates the system's theoretical foundation. Under acute stress (red arrow), speakers experience inhibitory control failure, shifting along the continuum from acrolectal (standard) to basilectal (creole) registers. As speech becomes more basilectal, ASR confidence (blue line) degrades below the usable threshold of 0.7. The ``Intervention Zone'' highlights TRIDENT's novel contribution: identifying calls where low transcription confidence coincides with high bio-acoustic distress, thereby converting a technical failure into a high-priority (Q1) routing signal.}
\label{fig:theoretical_model}
\end{figure}

This thesis rests on the psycholinguistic literature establishing that:
\begin{enumerate}
    \item Stress triggers cognitive load effects that impair executive function \cite{gollan2009}
    \item Impaired executive function leads to reduced inhibition of dominant language varieties \cite{green1998}
    \item For Caribbean speakers, dominant varieties include basilectal forms underrepresented in ASR training \cite{patrick1999, madden2025}
    \item Stress simultaneously elevates bio-acoustic markers (F0, intensity) that can be detected independently of speech content \cite{vanpuyvelde2018}
\end{enumerate}

The logical conclusion: when ASR confidence drops and bio-acoustic distress rises, the system has detected a caller in genuine crisis whose speech has shifted beyond standard recognition capabilities. This combination should trigger immediate human review---not because the system has failed, but because it has successfully identified a caller who needs human attention most.
