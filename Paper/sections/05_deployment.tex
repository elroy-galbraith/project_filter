\section{Deployment Considerations}

\subsection{Operational Context: Supporting Protocol Application}

TRIDENT integrates with existing emergency dispatch workflows to support standardized triage protocols---ESI for routine operations, START for mass casualty incidents. \textbf{Day-to-day (ESI context):} TRIDENT processes incoming calls to extract structured entities (location, mechanism, clinical indicators) and assigns queue priority. Dispatchers apply ESI to determine clinical acuity level (1--5) and appropriate response. \textbf{Mass casualty events (START context):} During hurricanes or earthquakes, TRIDENT's queue prioritization manages call surges when volume exceeds dispatcher capacity, enabling rapid caller sorting even when transcription quality degrades. \textbf{Key principle:} TRIDENT determines which calls dispatchers see first and what structured information they receive; clinical triage decisions remain with trained professionals applying Ministry of Health protocols.

\subsection{Primary Deployment: Surge Queue Prioritization}

TRIDENT's greatest value emerges during \textbf{disaster surge conditions}---hurricanes, earthquakes, floods---when call volume exceeds dispatcher capacity and callers must wait in queue. TRIDENT's processing latency (45--60 seconds on edge hardware) precludes real-time transcription, but surge queues provide ideal operational context.

\textbf{Operational flow:}
\begin{enumerate}
    \item Caller dials emergency services; all dispatchers engaged
    \item Caller enters queue and hears automated message requesting description
    \item Caller provides initial statement (15--30 seconds)
    \item TRIDENT processes audio while caller waits (45--60 seconds)
    \item Queue reordered by priority (Q1-IMMEDIATE through Q5-ROUTINE)
    \item Highest-priority call routes first when dispatcher becomes available
    \item Dispatcher receives transcription, extracted entities, and distress indicators to support ESI/START application
\end{enumerate}

\textbf{Why this context maximizes value:} Calls are waiting regardless---TRIDENT uses wait time productively. Queue prioritization ensures most critical callers reach dispatchers first. Extracted entities enable faster protocol application. Low ASR confidence flags alert dispatchers to potential dialect shift or audio quality issues before engagement.

This deployment model represents TRIDENT's primary design target. Caribbean emergency services face predictable annual surge events (hurricane season, June--November) where this capability would directly impact response effectiveness.

\subsection{Early Exit for Critical Cases}

To provide faster routing for clearly distressed callers, the system implements early exit when:
\begin{enumerate}
    \item \textbf{High Distress + Low Confidence:} If $D > 0.8$ and $C < 0.4$, route immediately to Q1-IMMEDIATE. This captures callers exhibiting extreme vocal stress whose speech has likely shifted to basilectal registers.

    \item \textbf{Extreme Distress:} If $D > 0.9$ regardless of confidence, route to Q1-IMMEDIATE.
\end{enumerate}

Under early exit, ASR and bio-acoustics complete in approximately 12 seconds (with bio-acoustic extraction parallel to transcription), reducing Time-to-Q1 from 55 seconds to 12 seconds for clearly distressed callers---a critical improvement for surge queue scenarios.

\subsection{Offline Operation}

All components operate without internet connectivity: Whisper model weights and Llama 3 stored locally, bio-acoustic analysis uses standard signal processing libraries, and queue logic implemented in local Python. This enables deployment at emergency coordination centers that may lose connectivity during disasters while maintaining local power (generator/battery backup). Offline capability ensures TRIDENT can support ESI/START protocol application precisely when infrastructure degradation makes accurate call processing most difficult.

\subsection{Integration with Existing Dispatch Systems}

TRIDENT operates as a \textbf{pre-processing layer} integrating with existing Computer-Aided Dispatch (CAD) systems. The system accepts audio streams, processes them through the three-layer architecture, and outputs structured data packages (queue priority, transcription with confidence, extracted entities, distress indicators) to CAD systems. Dispatchers receive calls in priority order and apply ESI or START protocols using TRIDENT's structured data and/or direct audio review. This requires no changes to clinical protocols---only familiarization with TRIDENT's output format.

\subsection{Hardware Requirements}

The complete system deploys on Raspberry Pi 5 (8GB RAM) or equivalent edge hardware:

\begin{table}[ht]
\centering
\begin{tabular}{@{}llll@{}}
\toprule
\textbf{Component} & \textbf{Model} & \textbf{Size} & \textbf{Inference Speed} \\ \midrule
ASR & Whisper Medium (INT4) & $\sim$400MB & $\sim$10s per 30s audio \\
NLP & Llama 3 8B (4-bit) & $\sim$4GB & 2-5 tokens/sec \\
Bio-acoustic & librosa + numpy & $<$50MB & Real-time \\ \bottomrule
\end{tabular}
\caption{Hardware requirements for edge deployment}
\label{tab:hardware}
\end{table}

Total system footprint: $\sim$4.5GB, well within Raspberry Pi 5 8GB capacity.
