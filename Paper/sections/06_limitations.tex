\section{Limitations and Future Work}

\subsection{Current Limitations}

\textbf{Validation gap (most critical).} This paper presents an architectural framework with theoretical grounding but limited empirical validation on real emergency calls. Performance claims are based on component evaluations and related literature rather than end-to-end system testing. The three-dimensional queue prioritization matrix has not been validated against expert dispatcher judgments.

\textbf{Protocol integration.} While TRIDENT is framed as supporting ESI and START protocols, the entity extraction schema and queue prioritization logic were developed independently of clinical stakeholder input. Full Ministry of Health integration requires validation that extracted entities map correctly to ESI decision points and that queue priorities align with operational workflows.

\textbf{Training data constraints.} Caribbean emergency speech corpora do not exist. ASR fine-tuning was performed on broadcast speech, which differs from emergency call acoustics in noise profiles, emotional content, and register distribution.

\textbf{Sex differences in F0 baseline.} Fundamental frequency is sexually dimorphic: male voices typically range 85--175 Hz while female voices range 165--270 Hz \cite{titze1989, traunmuller1995}. We mitigate this by prioritizing sex-normalized features (F0 coefficient of variation, jitter) over absolute F0 elevation in distress score calculation. Research confirms that stress manifests with ``striking parallels in men and women'' \cite{pisanski2018}---both sexes show increased pitch mean and variation under acute stress. However, residual bias risks remain: relaxed female speakers near upper baseline may contribute to elevated distress scores, while stressed male speakers with naturally low F0 may not contribute sufficiently. A validation study with sex-stratified analysis on Caribbean emergency calls is essential to calibrate population-appropriate thresholds and confirm normalized measures maintain sensitivity across demographics.

\textbf{Content indicator classification.} The Content Indicator Score depends on LLM classification quality. Caribbean creole expressions not well-represented in training data may be misclassified. Empirical evaluation of classification accuracy on Caribbean transcripts is needed, particularly for false negatives that could delay critical calls.

\textbf{Single-speaker assumption.} Multi-party calls are not handled. Speaker changes mid-call could confuse bio-acoustic analysis and entity extraction.

\textbf{Threshold sensitivity.} Multiple thresholds (ASR confidence 0.7, distress 0.5, content indicators 50) were selected based on literature but have not been rigorously optimized. Sensitivity analysis examining precision-recall tradeoffs is needed.

\subsection{Future Work}

\textbf{Clinical stakeholder collaboration.} Partnership with Caribbean emergency services to validate TRIDENT's utility in real dispatch workflows, including observation studies of current ESI/START challenges, dispatcher feedback on extracted entity usefulness, and iterative schema refinement based on clinical input.

\textbf{Caribbean Emergency Speech Corpus.} A dedicated corpus combining Caribbean-accented speech with emergency domain content and stress annotations is critical. We are exploring \textit{VoicefallJA}, a gamified speech elicitation platform designed to collect stressed Caribbean speech through game-induced cognitive load rather than acted performance. The Progressive Web App targets 100--300 speakers via church network distribution, with Q2--Q3 2026 data collection. However, game-induced stress differs fundamentally from genuine emergency distress; this approach should be viewed as a stepping stone toward real-call annotation under appropriate ethical frameworks, not a replacement.

\textbf{Empirical validation.} End-to-end evaluation with emergency dispatch professionals assessing whether TRIDENT's queue prioritization aligns with expert judgment, including sex-stratified analysis of bio-acoustic accuracy and entity extraction accuracy on Caribbean creole transcripts.

\textbf{Ablation studies.} Quantifying the marginal contribution of each architectural component (bio-acoustic analysis, content indicators, Caribbean-tuned ASR).

\textbf{Sex-adaptive distress detection.} Implementing within-call F0 change detection rather than absolute thresholds, and ensemble approaches combining multiple normalization strategies.
