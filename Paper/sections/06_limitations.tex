\section{Limitations and Future Work}

\subsection{Current Limitations}

\textbf{Validation gap (most critical).} This paper presents an architectural framework with theoretical grounding but limited empirical validation on real emergency calls. Performance claims for each layer are based on component evaluations and related literature rather than end-to-end system testing. The three-dimensional queue prioritization matrix (ASR confidence $\times$ distress $\times$ content indicators) is theoretically motivated but has not been validated against expert dispatcher judgments.

\textbf{Protocol integration.} While this paper frames TRIDENT as a dispatcher-support system for ESI and START protocol application, the entity extraction schema and queue prioritization logic were developed independently of clinical stakeholder input. Full integration with Ministry of Health workflows would require:
\begin{itemize}
    \item Validation that extracted entities map correctly to ESI decision points A--D
    \item Confirmation that queue priority levels align with operational dispatcher workflows
    \item Assessment of whether bio-acoustic distress indicators provide actionable information beyond what dispatchers already perceive
    \item Training material development for dispatcher familiarization with TRIDENT outputs
\end{itemize}

\noindent This clinical integration work represents essential future collaboration with Caribbean emergency services professionals. The current paper establishes technical feasibility; operational validation requires partnership with the health ministries whose protocols TRIDENT aims to support.

\textbf{Training data constraints.} Caribbean emergency speech corpora do not exist. ASR fine-tuning was performed on broadcast speech, which differs significantly from emergency call acoustics in noise profiles, emotional content, and register distribution. The gap between training domain (broadcast) and deployment domain (emergency calls) may introduce systematic errors not captured in current evaluation.

\textbf{Bio-acoustic threshold calibration.} Distress detection thresholds are derived from literature on non-Caribbean, predominantly Western populations. Baseline vocal characteristics may vary across Caribbean demographics, requiring population-specific calibration.

\textbf{Sex differences in F0 baseline.} Fundamental frequency is sexually dimorphic: male voices typically range 85--175 Hz while female voices range 165--270 Hz \cite{titze1989}. We partially address this through architectural choices:
\begin{itemize}
    \item Prioritizing sex-normalized features: F0 coefficient of variation ($CV = \sigma_{F0} / \mu_{F0}$) and jitter, which measure relative instability rather than absolute values
    \item Weighting normalized features (CV: 0.35, jitter: 0.15) more heavily than absolute F0 elevation (0.30)
\end{itemize}

\noindent However, residual bias likely remains. Research confirms that stress manifests with ``striking parallels in men and women'' \cite{pisanski2018}---both sexes show increased pitch mean and variation under stress---but our reliance on any absolute F0 component creates risk:
\begin{itemize}
    \item \textbf{False positive risk:} A relaxed female speaker near the upper baseline range may contribute to elevated distress scores
    \item \textbf{False negative risk:} A stressed male speaker with naturally low F0 may not contribute sufficiently to the pitch component
\end{itemize}

\noindent A validation study with sex-stratified analysis is essential to quantify this bias and determine whether further threshold adjustment or feature re-weighting is required.

\textbf{Content indicator classification.} The Content Indicator Score depends on LLM classification quality. While leveraging Llama 3's semantic understanding avoids brittle keyword matching, it introduces new failure modes:
\begin{itemize}
    \item Classification errors propagate deterministically to queue priority
    \item Caribbean creole expressions not well-represented in LLM training data may be misclassified
    \item The model may fail to recognize culturally-specific threat indicators or landmarks
\end{itemize}

\noindent Empirical evaluation of classification accuracy on Caribbean emergency transcripts is needed, with particular attention to false negatives (urgent content classified as non-urgent) that could delay dispatcher attention to critical calls.

\textbf{Single-speaker assumption.} The current architecture assumes single-speaker input. Multi-party calls, common in emergencies (``put your mother on the phone''), are not handled. Speaker changes mid-call could confuse bio-acoustic analysis and entity extraction continuity.

\textbf{Threshold sensitivity.} Multiple thresholds govern system behavior: ASR confidence (0.7), distress score (0.5), and content indicators (50). These values were selected based on literature and initial calibration but have not been rigorously optimized. Sensitivity analysis examining system performance across threshold combinations is needed to understand precision-recall tradeoffs for each queue priority level.

\subsection{Future Work}

\textbf{Clinical stakeholder collaboration.} The most important next step is partnership with Caribbean emergency services to validate TRIDENT's utility in real dispatch workflows. This includes:
\begin{itemize}
    \item Observation studies of current ESI/START application challenges
    \item Dispatcher feedback on extracted entity usefulness and queue priority alignment
    \item Iterative refinement of the entity extraction schema based on clinical input
    \item Development of dispatcher training materials for TRIDENT integration
\end{itemize}

\textbf{Caribbean Emergency Speech Corpus.} A critical enabler for future progress is a dedicated corpus combining Caribbean-accented speech with emergency domain content and stress annotations. We are exploring partnerships with Caribbean emergency services to develop such a resource, with appropriate privacy protections and community consent.

\textbf{Empirical validation.} End-to-end evaluation with emergency dispatch professionals assessing whether TRIDENT's queue prioritization aligns with expert judgment. This should include:
\begin{itemize}
    \item Comparison of three-dimensional prioritization against two-dimensional (confidence $\times$ distress) baseline
    \item Sex-stratified analysis of bio-acoustic distress detection accuracy
    \item Assessment of entity extraction accuracy on Caribbean creole transcripts
    \item Measurement of dispatcher efficiency gains (if any) when using TRIDENT outputs
\end{itemize}

\textbf{Ablation studies.} Rigorous testing to quantify the contribution of each architectural component:
\begin{itemize}
    \item Does bio-acoustic analysis improve queue prioritization over ASR-only approaches?
    \item Do content indicators catch urgent calls missed by distress detection alone?
    \item What is the marginal value of Caribbean-tuned ASR versus off-the-shelf Whisper?
\end{itemize}

\textbf{Sex-adaptive distress detection.} Implementing and validating approaches to further reduce sex bias:
\begin{itemize}
    \item Within-call F0 \textit{change} detection rather than absolute thresholds
    \item Automatic speaker characteristic estimation for threshold adaptation
    \item Ensemble approaches combining multiple normalization strategies
\end{itemize}

\textbf{Dialect density estimation.} Augmenting the system with automatic estimation of creole feature density, providing dispatchers with guidance on expected communication challenges and informing decisions about when to rely on extracted text versus direct audio review.

\textbf{Multilingual extension.} Caribbean emergency services handle calls in English, Spanish, French, Dutch, and various creoles. Extending the architecture to multilingual operation would significantly expand impact, though each language introduces its own ASR adaptation and entity extraction challenges.

\textbf{Edge deployment optimization.} While the architecture is designed for offline operation, current latency profiles (45--60 seconds per call) limit real-time applicability. Optimization for edge hardware (Raspberry Pi, embedded GPU) would enable faster queue prioritization at emergency coordination centers operating with degraded connectivity.