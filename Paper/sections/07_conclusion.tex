\section{Conclusion}

TRIDENT presents a defensive architecture for Caribbean emergency speech processing that treats ASR limitations not as failures to be eliminated but as signals to be incorporated into triage logic. By combining accent-adapted speech recognition, local NLP extraction, and bio-acoustic distress detection, the system maintains functionality across a range of conditions---including the high-stress, dialect-shifted speech most likely to defeat traditional ASR approaches.

The key insight is that low ASR confidence combined with high vocal distress is not a system failure but a system feature: the signature of a caller in genuine crisis whose speech patterns have shifted beyond the reach of standard recognition. Routing these calls to priority human review ensures that the most vulnerable callers receive the most urgent attention.

We hope this architectural framework contributes to more equitable emergency AI systems---not just for Caribbean populations, but for the billions of speakers worldwide whose accents and dialects remain underserved by current speech technology.
