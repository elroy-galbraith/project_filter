\section{Conclusion}

TRIDENT presents a dispatcher-support architecture that ensures Caribbean-accented emergency callers receive equitable access to ESI and START triage protocols. By combining accent-adapted speech recognition, local NLP entity extraction, and bio-acoustic distress detection, the system empowers dispatchers to apply established protocols even when automated transcription fails.

The architecture operationalizes two complementary insights established in Section~\ref{sec:key_insights}: that ASR uncertainty combined with vocal distress signals priority callers requiring human attention, and that calm delivery of urgent content must not delay dispatcher response. These insights drive the three-dimensional queue prioritization matrix that routes calls based on confidence, content, and concern signals.

Critically, TRIDENT respects the clinical authority of established protocols. The system determines which calls dispatchers see first and provides structured information to support rapid protocol application---but triage decisions remain with trained human professionals. This design philosophy reflects a broader principle for emergency AI: technology should empower human expertise, not attempt to replace it.

We hope this architectural framework contributes to more equitable emergency services---not just for Caribbean populations, but for the billions of speakers worldwide whose accents and dialects remain underserved by current speech technology. When a caller dials for help, the system that answers should understand them. TRIDENT is a step toward that goal.