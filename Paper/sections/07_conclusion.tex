\section{Conclusion}

TRIDENT presents a dispatcher-support architecture for Caribbean emergency speech processing that treats ASR limitations not as failures to be hidden but as signals to inform queue prioritization. By combining accent-adapted speech recognition, local NLP entity extraction, and bio-acoustic distress detection, the system empowers dispatchers to apply established triage protocols---ESI for routine operations, START for mass casualty events---even when automated transcription fails.

The key insight is that low ASR confidence combined with high vocal distress signals a caller requiring priority human attention. Rather than representing system failure, this pattern is the acoustic signature of a caller in genuine crisis whose speech has shifted toward basilectal registers under stress. TRIDENT ensures these callers reach dispatchers first, enabling timely application of ESI or START protocols to those who need them most urgently.

A complementary insight drives the content analysis layer: trained professionals and composed bystanders may report critical emergencies without elevated vocal stress. By extracting clinical entities that map to ESI/START decision points, TRIDENT ensures that calm delivery of urgent content does not delay dispatcher attention.

Critically, TRIDENT respects the clinical authority of established protocols. The system determines which calls dispatchers see first and provides structured information to support rapid protocol application---but triage decisions remain with trained human professionals. This design philosophy reflects a broader principle for emergency AI: technology should empower human expertise, not attempt to replace it.

We hope this architectural framework contributes to more equitable emergency services---not just for Caribbean populations, but for the billions of speakers worldwide whose accents and dialects remain underserved by current speech technology. When a caller dials for help, the system that answers should understand them. TRIDENT is a step toward that goal.