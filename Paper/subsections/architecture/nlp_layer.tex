\subsection{Layer 2: Local NLP Entity Extraction}

When ASR produces usable transcription (confidence $\geq$ 0.7), the NLP layer extracts structured emergency information using Llama 3 8B running locally via Ollama. The extraction schema targets four entity types critical for emergency dispatch:

\begin{itemize}
    \item \textbf{LOCATION:} Street addresses, landmarks, geographic references
    \item \textbf{HAZARD:} Emergency type (fire, flood, medical, violence, etc.)
    \item \textbf{PERSONS:} Number of people involved, injuries mentioned
    \item \textbf{URGENCY:} Temporal markers (``right now,'' ``hurry,'' breathing patterns)
\end{itemize}

\textbf{Handling Garbled Input:} A critical design question is how the NLP layer behaves when ASR produces low-quality transcriptions. We address this through confidence-aware prompting:

\begin{verbatim}
SYSTEM: You are extracting emergency information from a
speech transcript. The transcription confidence is
{confidence_score}.

If confidence is below 0.7, the transcript may contain errors.
Extract what you can, but:
1. Mark uncertain extractions with [UNCERTAIN]
2. Do not hallucinate or guess missing information
3. Prioritize extracting any recognizable location names
4. Note phonetically similar alternatives for garbled terms

TRANSCRIPT: {asr_output}
\end{verbatim}

\textbf{Example of garbled transcript handling:}

\begin{table}[ht]
\centering
\small
\begin{tabular}{@{}p{0.45\textwidth}p{0.45\textwidth}@{}}
\toprule
\textbf{ASR Output (confidence=0.52)} & \textbf{NLP Extraction} \\ \midrule
``mi house a bun down pan [unintelligible] road near di gas station'' &
LOCATION: ``[UNCERTAIN] road, near gas station''; HAZARD: ``fire (house burning)''; PERSONS: ``unknown''; URGENCY: ``high'' \\ \bottomrule
\end{tabular}
\caption{Example of NLP extraction from low-confidence ASR output}
\label{tab:garbled_example}
\end{table}

When confidence is very low ($<$0.4), the NLP layer produces minimal structured output and flags the call for immediate human review, relying on the bio-acoustic layer to provide triage guidance.

\subsubsection{Content Severity Scoring}

Beyond entity extraction, the NLP layer computes a \textbf{Content Severity Score} ($S_c \in [0,100]$) that quantifies the urgency implied by the \textit{semantic content} of the transcript, independent of how the caller sounds. This addresses a critical gap: a trained first responder or composed bystander may report a mass casualty event in a calm voice, producing low bio-acoustic distress despite extremely urgent content.

\textbf{Classification-Based Approach.} Rather than brittle keyword matching, we leverage the LLM's semantic understanding to classify transcript content along four dimensions. This approach offers critical advantages for Caribbean speech: the model can recognize that ``mi granmodda drop dung an she nah move'' conveys the same urgency as ``my grandmother collapsed and she's not moving'' without requiring an exhaustive enumeration of creole variants. The LLM also handles negation (``no one is trapped''), indirect references (``she nine months pregnant'' $\rightarrow$ vulnerable), and context-dependent interpretation that keyword matching cannot capture.

The LLM outputs structured classifications according to the following schema:

\begin{verbatim}
{
  "hazard_category": "violent_crime" | "medical" | "fire" | 
                     "flood" | "traffic" | "infrastructure" | "other",
  "life_threat_level": "imminent" | "potential" | "none",
  "vulnerable_population": true | false,
  "situation_status": "escalating" | "stable" | "resolved",
  "persons_affected": <integer>
}
\end{verbatim}

A deterministic scoring function then maps these classifications to the Content Severity Score, separating the flexibility of neural language understanding from the interpretability of rule-based scoring:

\begin{equation}
S_c = \min\left(100,\ S_{\text{hazard}} + S_{\text{threat}} + S_{\text{vuln}} + S_{\text{scale}}\right)
\label{eq:content_severity}
\end{equation}

\textbf{Component 1: Hazard Category ($S_{\text{hazard}}$).} Different emergency types carry inherent urgency levels:

\begin{table}[ht]
\centering
\small
\begin{tabular}{@{}ll@{}}
\toprule
\textbf{Hazard Category} & \textbf{Score} \\ \midrule
\texttt{violent\_crime} & 30 \\
\texttt{medical} & 25 \\
\texttt{fire} & 25 \\
\texttt{flood} & 20 \\
\texttt{traffic} & 15 \\
\texttt{infrastructure} & 10 \\
\texttt{other} & 5 \\ \bottomrule
\end{tabular}
\caption{Hazard category severity scores}
\label{tab:hazard_weights}
\end{table}

\textbf{Component 2: Life-Threat Level ($S_{\text{threat}}$).} The LLM classifies the immediacy of danger to life based on semantic understanding of the full transcript context:

\begin{itemize}
    \item \texttt{imminent}: Active, immediate threat to life (trapped, not breathing, active violence, drowning) $\rightarrow$ +30
    \item \texttt{potential}: Situation could become life-threatening (injuries, spreading fire, chest pain) $\rightarrow$ +15
    \item \texttt{none}: No apparent threat to life $\rightarrow$ +0
\end{itemize}

\textbf{Component 3: Vulnerable Population ($S_{\text{vuln}}$).} Boolean classification indicating presence of children, elderly, pregnant individuals, or persons with disabilities. If \texttt{true} $\rightarrow$ +15. This reflects both ethical prioritization and reduced self-rescue capacity.

\textbf{Component 4: Scale and Escalation ($S_{\text{scale}}$).} Combines two factors:
\begin{itemize}
    \item \texttt{persons\_affected}: +5 per person, capped at +20
    \item \texttt{situation\_status = "escalating"}: +10 (fire spreading, water rising, more vehicles involved)
\end{itemize}

\textbf{Example severity calculations:}

\begin{table}[ht]
\centering
\small
\begin{tabular}{@{}p{0.38\textwidth}p{0.35\textwidth}c@{}}
\toprule
\textbf{Transcript} & \textbf{LLM Classification} & \textbf{$S_c$} \\ \midrule
``Pothole on Nelson Street'' & 
\texttt{infrastructure, none, false, stable, 0} & 10 \\[0.5em]
``Car accident, one person injured'' & 
\texttt{traffic, potential, false, stable, 1} & 35 \\[0.5em]
``House fire, spreading to neighbor's yard'' & 
\texttt{fire, potential, false, escalating, 0} & 50 \\[0.5em]
``Mi granmodda drop dung, she nah breathe'' & 
\texttt{medical, imminent, true, stable, 1} & 75 \\[0.5em]
``Pickney dem trap inna di fire'' & 
\texttt{fire, imminent, true, stable, 2+} & 80 \\ \bottomrule
\end{tabular}
\caption{Content severity scoring via LLM classification. The model's semantic understanding captures urgency from both standard English and Caribbean creole variants without explicit keyword enumeration.}
\label{tab:severity_examples}
\end{table}

The Content Severity Score provides the third dimension of the triage decision matrix (Section~\ref{sec:triage_matrix}), ensuring that semantically urgent calls receive priority routing even when vocal distress markers are low.