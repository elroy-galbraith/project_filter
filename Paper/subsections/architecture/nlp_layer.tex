\subsection{Layer 2: Local NLP Entity Extraction}

When ASR produces usable transcription (confidence $\geq$ 0.7), the NLP layer extracts structured emergency information using Llama 3 8B running locally via Ollama. The extraction schema targets entity types that map directly to ESI and START triage protocol decision points.

\subsubsection{Entity Extraction Schema}

The schema targets four entity categories:
\begin{itemize}
    \item \textbf{LOCATION:} Street addresses, landmarks, geographic references
    \item \textbf{MECHANISM/HAZARD:} Emergency type (fire, flood, medical, violence, traffic)
    \item \textbf{CLINICAL INDICATORS:} Breathing status, consciousness, bleeding, mobility
    \item \textbf{SCALE:} Number of people involved, vulnerable populations
\end{itemize}

\subsubsection{Mapping to Triage Protocols}

TRIDENT entities support ESI and START protocol application. For ESI, extracted entities inform the four decision points: Point A (lifesaving intervention) captures "not breathing," "choking," "unresponsive"; Point B (high-risk situation) captures mechanism of injury and altered status; Point C (resource needs) uses hazard type and complexity; Point D (vital signs) uses reported vitals and distress indicators \cite{esi_handbook}.

For mass casualty events using START, entities support rapid sorting: GREEN captures "walking," "minor injuries"; YELLOW captures "injured but stable," "conscious"; RED captures "trapped," "not breathing," "heavy bleeding"; BLACK captures cessation indicators.

\begin{table}[ht]
\centering
\small
\begin{tabular}{@{}p{2.5cm}p{4cm}p{5.5cm}@{}}
\toprule
\textbf{Protocol} & \textbf{Decision Point} & \textbf{Example Extraction Target} \\ \midrule
ESI Level 1 & Immediate lifesaving intervention? & ``not breathing,'' ``choking,'' ``heavy bleeding,'' ``unresponsive'' \\[0.5em]
START RED & Not walking, breathing issues & ``trapped,'' ``not breathing,'' ``unresponsive,'' ``heavy bleeding'' \\ \bottomrule
\end{tabular}
\caption{Example entity extraction targets supporting ESI and START protocols. Full protocol mappings detailed in extended version.}
\label{tab:protocol_mapping_simplified}
\end{table}

\subsubsection{Handling Garbled Input}

The NLP layer handles low-quality transcriptions through confidence-aware prompting. When ASR confidence is below 0.7, the system instructs the LLM to mark uncertain extractions, avoid hallucination, prioritize location extraction, and note phonetically similar alternatives. When confidence is very low ($<$0.4), minimal structured output is produced and the call is flagged for immediate human review.

\subsubsection{Content Indicator Scoring}

The NLP layer computes a \textbf{Content Indicator Score} ($S_c \in [0,100]$) quantifying urgency implied by semantic content, independent of how the caller sounds. This addresses a critical gap: a trained first responder may report a mass casualty event calmly, producing low bio-acoustic distress despite extremely urgent content. Without content analysis, such calls would be deprioritized.

Rather than keyword matching, we leverage the LLM's semantic understanding to classify transcript content. This approach handles Caribbean creole variants (``mi granmodda drop dung an she nah move'' conveys the same urgency as ``my grandmother collapsed and she's not moving''), negation, and indirect references.

The LLM outputs structured classifications:
\begin{verbatim}
{
  "hazard_category": "violent_crime" | "medical" | "fire" |
                     "flood" | "traffic" | "infrastructure" | "other",
  "life_threat_level": "imminent" | "potential" | "none",
  "vulnerable_population": true | false,
  "situation_status": "escalating" | "stable" | "resolved",
  "persons_affected": <integer>
}
\end{verbatim}

A deterministic function maps classifications to the score:
\begin{equation}
S_c = \min\left(100,\ S_{\text{hazard}} + S_{\text{threat}} + S_{\text{vuln}} + S_{\text{scale}}\right)
\label{eq:content_severity}
\end{equation}

\textbf{Scoring components:} Hazard category weights range from 30 (violent crime) to 5 (other). Life-threat level contributes +30 (imminent), +15 (potential), or +0 (none). Vulnerable population adds +15. Scale combines persons affected (+5 per person, capped at +20) and escalation status (+10 if escalating).

\textbf{Example calculations:}

\begin{table}[ht]
\centering
\small
\begin{tabular}{@{}p{0.40\textwidth}p{0.35\textwidth}c@{}}
\toprule
\textbf{Transcript} & \textbf{Classification} & \textbf{$S_c$} \\ \midrule
``Pothole on Nelson Street'' &
infrastructure, none, false, stable, 0 & 10 \\[0.5em]
``House fire, spreading to neighbor's yard'' &
fire, potential, false, escalating, 0 & 50 \\[0.5em]
``Pickney dem trap inna di fire'' &
fire, imminent, true, stable, 2+ & 80 \\ \bottomrule
\end{tabular}
\caption{Content indicator scoring via LLM classification. Semantic understanding captures urgency from Caribbean creole variants. High scores elevate queue priority; clinical triage remains with dispatchers.}
\label{tab:severity_examples}
\end{table}

The Content Indicator Score feeds into queue prioritization (Section~\ref{sec:queue_prioritization}), ensuring semantically urgent calls reach dispatchers promptly even when vocal distress markers are absent. Weights are tunable parameters that should be calibrated with local emergency services to reflect institutional priorities and regional hazard profiles.
