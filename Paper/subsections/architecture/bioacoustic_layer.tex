\subsection{Layer 3: Bio-Acoustic Distress Detection}

The bio-acoustic layer operates on raw audio, independent of ASR success, extracting features correlated with psychological distress. Based on the vocal stress literature \cite{schmalz2025, vanpuyvelde2018, veiga2025}, we focus on features that capture physiological arousal through vocal production changes.

\subsubsection{Feature Extraction}

Using librosa, we extract the following acoustic features:

\begin{enumerate}
    \item \textbf{Fundamental Frequency (F0):} Mean pitch extracted via autocorrelation method
    \begin{itemize}
        \item Typical baseline: 85--180 Hz (male), 165--255 Hz (female) \cite{titze1989}
        \item Stress indicator: Elevation above speaker baseline
    \end{itemize}

    \item \textbf{F0 Coefficient of Variation (CV):} Pitch instability measure
    \begin{itemize}
        \item Computed as $CV = \sigma_{F0} / \mu_{F0}$
        \item Normalizes for baseline differences across speakers
        \item Stress indicator: $CV > 0.3$ suggests vocal instability
    \end{itemize}

    \item \textbf{Energy (RMS amplitude):} Mean intensity across utterance
    \begin{itemize}
        \item Normalized to 0--1 scale relative to recording gain
        \item Stress indicator: Elevated intensity during distress vocalizations
    \end{itemize}

    \item \textbf{Jitter:} Cycle-to-cycle variation in F0 period
    \begin{itemize}
        \item Relatively independent of prosodic patterns \cite{vanpuyvelde2018}
        \item Pathology threshold: $>$1.04\% \cite{boersma2013}
    \end{itemize}
\end{enumerate}

\subsubsection{Distress Score Calculation}

The distress score combines multiple acoustic indicators into a composite metric. We weight features according to their documented reliability and sex-independence:

\begin{align}
D &= w_{\text{pitch}} \cdot P + w_{\text{var}} \cdot V + w_{\text{energy}} \cdot E + w_{\text{jitter}} \cdot J
\label{eq:distress}
\end{align}

\noindent where:

\begin{align}
P &= \min\left(1.0, \max\left(0, \frac{\bar{F_0} - 180}{120}\right)\right) & \text{(pitch elevation)} \\
V &= \min\left(1.0, \frac{CV_{F0}}{0.5}\right) & \text{(pitch instability)} \\
E &= \min\left(1.0, \frac{\bar{E}}{0.1}\right) & \text{(energy)} \\
J &= \min\left(1.0, \frac{\text{jitter}}{0.02}\right) & \text{(perturbation)}
\end{align}

The weights reflect relative reliability from the literature:
\begin{itemize}
    \item $w_{\text{pitch}} = 0.30$ --- F0 elevation is the most consistent stress marker but is sex-dependent
    \item $w_{\text{var}} = 0.35$ --- F0 coefficient of variation is sex-normalized and robust
    \item $w_{\text{energy}} = 0.20$ --- intensity elevation accompanies distress
    \item $w_{\text{jitter}} = 0.15$ --- perturbation measures are prosody-independent
\end{itemize}

\subsubsection{Threshold Classification}

\begin{itemize}
    \item \textbf{High Distress:} $D > 0.5$
    \item \textbf{Low Distress:} $D \leq 0.5$
\end{itemize}

These thresholds are calibrated against Van Puyvelde et al.'s \cite{vanpuyvelde2018} findings on vocal markers in emergency versus baseline speech.

\textbf{Note on sex differences:} The distress score prioritizes sex-normalized features (CV, jitter) over absolute F0 elevation to mitigate the substantial baseline differences between male (85--175 Hz) and female (165--270 Hz) speakers. See Section~\ref{sec:sex_limitations} for detailed discussion of remaining bias risks.