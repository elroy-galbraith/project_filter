\subsection{The Complementarity Principle}

The theoretical foundation for our multi-layer design rests on what we term the \textbf{Complementarity Principle}: the three triage dimensions capture distinct failure modes and urgency signals that compensate for each other's blind spots.

\textbf{Dimension 1: ASR Confidence.} The conditions that degrade ASR performance (high stress, code-switching to basilect, environmental noise) are precisely the conditions that often accompany genuine emergencies. Low confidence is not merely a technical limitation---it correlates with caller distress.

\textbf{Dimension 2: Bio-Acoustic Distress.} Vocal stress markers (elevated pitch, intensity, instability) provide a parallel assessment channel that operates on raw audio, independent of transcription success. A caller whose speech is entirely unintelligible to ASR will still produce detectable distress signals.

\textbf{Dimension 3: Content Severity.} Semantic analysis of transcript content captures urgency that vocal characteristics may miss. Trained professionals, repeat callers, and composed bystanders often report critical emergencies without elevated vocal stress---their calm delivery masks the urgency that only content analysis reveals.

This creates a robust triage space with complementary coverage:

\begin{itemize}
    \item \textbf{High Confidence + Low Distress + Low Severity:} Routine call, automated processing appropriate
    \item \textbf{High Confidence + Low Distress + High Severity:} The composed reporter---urgent content from a calm caller requires priority handling despite absent vocal stress markers
    \item \textbf{High Confidence + High Distress + Low Severity:} Anxious caller, minor issue---human review to de-escalate
    \item \textbf{High Confidence + High Distress + High Severity:} Confirmed emergency, all signals aligned
    \item \textbf{Low Confidence + Low Distress + Low Severity:} Likely technical issue, re-prompt or review
    \item \textbf{Low Confidence + Low Distress + High Severity:} Garbled but fragments suggest urgency---human ears needed
    \item \textbf{Low Confidence + High Distress + Low Severity:} Distressed caller, unintelligible speech---priority human review
    \item \textbf{Low Confidence + High Distress + High Severity:} Maximum urgency---all indicators elevated, immediate dispatch
\end{itemize}

Two cells represent our key insights. The \textbf{Low Confidence + High Distress} cases (regardless of content) capture callers in crisis whose speech has shifted toward basilectal registers---we interpret ASR failure combined with vocal stress as valuable triage information rather than system failure. The \textbf{High Confidence + Low Distress + High Severity} cell captures the complementary blind spot: the trained first responder or composed bystander whose calm voice belies the urgency of their report. Together, these insights ensure that neither paralinguistic nor semantic signals alone determine routing.