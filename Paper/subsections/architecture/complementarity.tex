\subsection{The Complementarity Principle}

The theoretical foundation for our multi-layer design rests on what we term the \textbf{Complementarity Principle}: the three signal dimensions capture distinct failure modes and urgency indicators that compensate for each other's blind spots, ensuring dispatchers receive the most critical calls first regardless of which individual signal might fail.

\textbf{Dimension 1: Transcription Confidence.} The conditions that degrade ASR performance (high stress, code-switching to basilect, environmental noise) are precisely the conditions that often accompany genuine emergencies. Low confidence is not merely a technical limitation to be hidden---it correlates with caller distress and should elevate queue priority while flagging the call for direct audio review.

\textbf{Dimension 2: Content Indicators.} Semantic analysis of transcript content captures urgency that vocal characteristics may miss. Trained professionals, repeat callers, and composed bystanders often report critical emergencies without elevated vocal stress---their calm delivery masks the urgency that only content analysis reveals. When transcription confidence is high, extracted entities map directly to ESI/START decision points.

\textbf{Dimension 3: Bio-Acoustic Distress.} Vocal stress markers (elevated pitch, intensity, instability) provide a parallel assessment channel that operates on raw audio, independent of transcription success. A caller whose speech is entirely unintelligible to ASR will still produce detectable distress signals. This dimension captures information not currently used by ESI or START protocols, representing TRIDENT's novel contribution to dispatcher awareness.

This creates a robust prioritization space with complementary coverage:

\textbf{Dimensional ordering.} The three dimensions are evaluated in deliberate sequence: \emph{Confidence}, \emph{Content}, \emph{Concern}. This ordering reflects operational logic: (1) \emph{Can we understand the caller?}---ASR confidence determines whether transcription is reliable enough for downstream analysis; (2) \emph{What is being reported?}---semantic content establishes the substance of the emergency; (3) \emph{How distressed does the caller sound?}---bio-acoustic indicators validate and can elevate priority, but do not override content. This sequence ensures that a composed professional reporting a mass casualty event receives appropriate priority based on content, while a highly distressed caller reporting a minor issue is not over-prioritized based on vocal expression alone.

\begin{itemize}
    \item \textbf{High Confidence + Low Content + Low Concern:} Routine call; dispatcher applies ESI using extracted entities at normal pace
    
    \item \textbf{High Confidence + High Content + Low Concern:} The composed reporter---urgent content from a calm caller requires elevated queue position; dispatcher reviews entities and applies ESI, likely assigning ESI-2 or ESI-3
    
    \item \textbf{High Confidence + Low Content + High Concern:} Anxious caller, possibly minor issue---dispatcher assesses whether distress reflects emergency or anxiety
    
    \item \textbf{High Confidence + High Content + High Concern:} All signals aligned; immediate queue position for rapid ESI/START application
    
    \item \textbf{Low Confidence + Low Content + Low Concern:} Likely technical issue; dispatcher reviews audio quality before processing
    
    \item \textbf{Low Confidence + High Content + Low Concern:} Garbled but fragments suggest urgency---elevated priority; dispatcher listens directly
    
    \item \textbf{Low Confidence + Low Content + High Concern:} Distressed caller with unintelligible speech---immediate priority; dispatcher listens and applies protocol based on direct assessment
    
    \item \textbf{Low Confidence + High Content + High Concern:} Maximum queue priority---all indicators suggest crisis; immediate dispatcher attention
\end{itemize}

Two cells represent our key insights. The \textbf{High Confidence + High Content + Low Concern} cell captures callers whose semantic content demands urgent attention despite calm delivery: the trained first responder, medical professional, or composed bystander whose measured voice belies the severity of their report. The \textbf{Low Confidence + Low Content + High Concern} cases capture the complementary pattern---callers in crisis whose speech has shifted toward basilectal registers, where ASR failure combined with vocal stress becomes valuable prioritization information rather than system failure.

Together, these insights ensure that neither semantic nor paralinguistic signals alone determine queue position---and that clinical triage decisions remain with trained dispatchers who can assess the full context of each call.
