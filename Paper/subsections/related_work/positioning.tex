\subsection{Summary: Positioning Our Contribution}

The literature reveals a clear opportunity space for Caribbean emergency services. Existing dispatch AI systems---while valuable as dispatcher-support tools---uniformly exhibit limitations that preclude effective deployment in Caribbean contexts:

\begin{enumerate}
    \item \textbf{Cloud dependency with accent-agnostic ASR:} Systems like ECA and Corti rely on commercial cloud APIs (Google Speech-to-Text, etc.) that show documented performance degradation on non-standard English varieties, with no adaptation for Caribbean accents or creole continua.
    
    \item \textbf{Text-only analysis:} Current approaches process only transcribed text, ignoring paralinguistic stress signals that may indicate caller distress even when words are unclear or mistranscribed.
    
    \item \textbf{Dialect blindness:} No existing system accounts for stress-induced register shifting---the phenomenon whereby speakers under acute stress revert toward basilectal (more creole-heavy) varieties, precisely when accurate transcription matters most.
    
    \item \textbf{Infrastructure fragility:} Cloud-dependent architectures fail during the communication infrastructure degradation that commonly accompanies hurricanes, earthquakes, and floods---the disasters that generate emergency call surges.
\end{enumerate}

\textbf{TRIDENT's contribution} is a dispatcher-support architecture that addresses each gap while respecting the clinical authority of established triage protocols:

\begin{itemize}
    \item \textbf{Caribbean-adapted ASR:} Fine-tuned Whisper models provide the transcription accuracy that makes downstream entity extraction viable for Caribbean speech varieties.
    
    \item \textbf{Structured entity extraction:} Local Llama 3-based NLP extracts clinical indicators needed for ESI/START application---location, mechanism of injury, breathing status, vulnerable populations---operating without internet connectivity.
    
    \item \textbf{Bio-acoustic distress detection:} A parallel signal pathway that functions even when ASR fails, transforming low transcription confidence from a system limitation into a queue prioritization feature that routes distressed callers to immediate human attention.
    
    \item \textbf{Offline operation:} Complete system deployment on edge hardware (Raspberry Pi 5) enables function during infrastructure failures when emergency services are most critical.
\end{itemize}

The result is the first dispatcher-support system designed specifically for Caribbean emergency services---not to make triage decisions, but to ensure that Caribbean-accented callers receive equitable access to the ESI and START protocols that their health ministries have adopted. TRIDENT empowers dispatchers with better information and intelligent queue prioritization; clinical judgment remains where it belongs---with trained human professionals.