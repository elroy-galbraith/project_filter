\subsection{Summary: Positioning Our Contribution}

TRIDENT addresses four critical gaps in existing emergency dispatch AI for Caribbean deployment:

\begin{itemize}
    \item \textbf{Caribbean-adapted ASR:} Fine-tuned Whisper models (informed by Madden et al.'s scaling laws) provide transcription accuracy for Caribbean speech varieties, enabling viable downstream entity extraction.

    \item \textbf{Multimodal distress detection:} Parallel bio-acoustic analysis provides a signal pathway that functions even when ASR fails, transforming low transcription confidence from a limitation into a queue prioritization feature.

    \item \textbf{Stress-aware design:} Accounts for stress-induced register shifting along the creole continuum---routing calls with elevated vocal distress and low ASR confidence to immediate human attention.

    \item \textbf{Offline operation:} Complete system deployment on edge hardware (Raspberry Pi 5) enables function during infrastructure failures when emergency services are most critical.
\end{itemize}

The result is the first dispatcher-support system designed specifically for Caribbean emergency services---not to make triage decisions, but to ensure Caribbean-accented callers receive equitable access to the ESI and START protocols that their health ministries have adopted. TRIDENT empowers dispatchers with better information and intelligent queue prioritization; clinical judgment remains with trained human professionals.
