\subsection{Edge Computing for Disaster Resilience}

The case for offline-capable emergency AI is made starkly by infrastructure failure during recent disasters. Hurricane Maria's impact on Puerto Rico saw 95\% of cell towers fail, with the entire island losing power and over 66\% of the population lacking potable water \cite{santosburgoa2020}. Communication infrastructure failure caused delays in mortality reporting and created substantial information vacuums, contributing to a disputed death toll ultimately estimated at approximately 3,000. Recovery required over 200 days for full power restoration.

Recent advances in model compression make edge deployment increasingly feasible. Quantization studies demonstrate that 4-bit (INT4) quantization reduces Whisper model size by 45-87\% with minimal WER degradation, and may actually reduce hallucinations by acting as a regularizer. Gondi and Pratap \cite{gondi2021} demonstrated that transformer-based ASR achieves real-time inference on Raspberry Pi hardware with PyTorch mobile optimization. For the NLP component, 4-bit quantized Llama 3 8B runs at 2-5 tokens per second on Raspberry Pi 5---too slow for real-time conversation but adequate for background entity extraction tasks.

A survey of edge technologies for disaster management identifies prediction, detection, response, and recovery phases where edge computing enables real-time processing without cloud dependency \cite{aboualola2023}. The survey specifically identifies a gap in offline-capable speech and language processing at the edge---precisely the capability our system provides. Pre-positioned edge computing resources at hospitals, shelters, and emergency coordination centers, loaded with Caribbean-tuned models, could maintain triage capability even during complete grid and network failure.
