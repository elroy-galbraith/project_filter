\subsection{Vocal Stress Detection}
\label{sec:vocal_stress}

The bio-acoustic layer builds on research establishing acoustic correlates of psychological stress. A systematic review of 38 studies found fundamental frequency (F0) as the most consistent stress marker, with 15 of 19 studies reporting significant mean F0 increases under stress \cite{schmalz2025}.

Research on emergency communications provides direct validation. Van Puyvelde et al. \cite{vanpuyvelde2018} analyzed real-life emergency recordings including cockpit voice recorders and 911 calls, documenting F0 increases from 123.9 Hz to 200.1 Hz during life-threatening emergencies---a 62\% increase. However, Deschamps-Berger et al. \cite{deschampsberger2021} found that while benchmark IEMOCAP data yielded 63\% emotion recognition accuracy, real emergency calls achieved only 45.6\%---a substantial domain shift. This finding reinforces our design decision to use bio-acoustic analysis as a triage signal routing high-distress calls to human dispatchers, rather than attempting fully automated classification.
