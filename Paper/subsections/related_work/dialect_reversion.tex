\subsection{Dialect Reversion Under Cognitive Load}

Psycholinguistic research establishes that for Caribbean speakers navigating the creole continuum---from basilect (most creole features) through mesolect to acrolect (Standard English)---maintaining acrolectal speech requires sustained executive function. The inhibitory control model establishes that non-target languages remain continuously active and must be suppressed through cognitive effort \cite{green1998}. Under high cognitive load, this inhibition fails, causing speakers to revert toward their dominant variety.

Patrick's \cite{patrick1999} sociolinguistic analysis of the Jamaican Creole continuum establishes that stress levels influence speakers' positioning on this spectrum, with most speakers being mesolectal under normal conditions but capable of shifting toward either pole. The implications for emergency services are significant: a professional who speaks Standard English at work may revert toward basilectal Patois when their house is flooding. Standard ASR systems will exhibit precisely the performance degradation documented in the accent gap literature at the moment when accurate recognition is most critical.
