\subsection{Dialect Reversion Under Cognitive Load}

A theoretical foundation for Caribbean-specific ASR in emergency contexts comes from psycholinguistic research on bilingual processing under stress. The inhibitory control model establishes that non-target languages remain continuously active and must be suppressed through cognitive effort \cite{green1998}. For Caribbean speakers navigating the creole continuum---from basilect (most creole features) through mesolect to acrolect (Standard English)---maintaining acrolectal speech requires sustained executive function.

\textbf{The creole continuum is not simply a stylistic choice but a dynamic system of linguistic control, modulated by cognitive load.} Research on cognitive load effects demonstrates that this inhibition fails under stress. Gollan and Ferreira \cite{gollan2009} found that under high cognitive load, bilingual speakers use significantly less intraclausal code-switching, instead reverting to monolingual chunks of their dominant language. Importantly, cognitive load also affects lexical access timing---Kroll et al. \cite{kroll2006} demonstrated that retrieval of L2 (non-dominant language) vocabulary slows significantly under dual-task conditions, providing a mechanism for stress-induced register shift.

Patrick's \cite{patrick1999} foundational sociolinguistic analysis of the Jamaican Creole continuum establishes that stress levels influence speakers' positioning on this spectrum, with most speakers being mesolectal in normal conditions but capable of shifting toward either pole.

The implications for emergency services are significant: a professional who speaks Standard English at work may revert toward basilectal Patois when their house is flooding. Standard ASR systems, trained predominantly on acrolectal varieties, will exhibit precisely the performance degradation documented in the accent gap literature at the moment when accurate recognition is most critical. Our system addresses this by fine-tuning on Caribbean broadcast data that includes mesolectal speech, and by providing bio-acoustic fallback when ASR confidence drops---which may itself serve as a proxy indicator for basilectal reversion.
