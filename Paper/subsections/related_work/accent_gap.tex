\subsection{The Accent Gap in Automatic Speech Recognition}

Modern ASR systems exhibit systematic performance degradation on non-standard English varieties. Koenecke et al. \cite{koenecke2020} evaluated five commercial ASR systems, finding word error rates averaged 0.35 for Black speakers compared to 0.19 for White speakers, with performance gaps traced to acoustic model limitations rather than language models.

Caribbean English remains especially underserved. Madden et al. \cite{madden2025} developed the first substantial Jamaican Patois corpus (42.58 hours) and derived scaling laws for Whisper performance. Pre-trained Whisper Large achieved 89\% WER on Patois, while fine-tuned Whisper Medium reduced this to 30\% WER. Critically, their scaling law (WER = 158.06 $\times$ M$^{-0.255}$ $\times$ D$^{-0.269}$) demonstrates that dataset increases yield greater gains than model scaling for underrepresented varieties, informing our choice of Whisper Medium with Caribbean-specific fine-tuning.
