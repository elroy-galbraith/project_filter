\subsection{AI-Assisted Emergency Dispatch}

Emergency services worldwide are exploring AI-powered speech recognition and natural language processing to improve call handling efficiency and triage accuracy. The Emergency Calls Assistant (ECA) framework represents current state-of-the-art, achieving 92.7\% accuracy in emergency classification using SVM with linear kernel on textual features \cite{attiah2025}. The system operates in two phases---speech-to-text conversion followed by NLP classification---and compares favorably against commercial platforms including RapidSOS, Corti, and AlertGO.

However, critical examination reveals systematic gaps in existing approaches. ECA relies on Google Cloud Speech-to-Text API with no offline capability or accent adaptation. The system processes only transcribed text, ignoring paralinguistic stress markers that may indicate caller distress even when words are unclear. Furthermore, due to privacy restrictions on real emergency recordings, ECA was trained on synthetic datasets---raising questions about generalization to actual crisis communications.

Clinical validation studies demonstrate AI's potential while highlighting implementation challenges. Blomberg et al. \cite{blomberg2019, blomberg2021} evaluated the Corti AI system for cardiac arrest detection, finding that the ML system achieved 84.1\% sensitivity compared to dispatchers' 72.5\%, with faster time-to-recognition (44 seconds versus 54 seconds median). However, a subsequent randomized clinical trial found no significant improvement in dispatcher recognition when supported by ML alerts---suggesting that human-AI teaming requires careful interface design beyond raw model performance.

A scoping review of 106 AI studies in prehospital emergency care identified underutilization of multimodal inputs as a key gap \cite{chee2023}. No reviewed system integrated audio-based stress detection with text classification---precisely the multi-layer approach we propose. The review also noted the absence of systems designed for infrastructure-independent operation, a critical limitation for disaster response scenarios.
