\subsection{AI-Assisted Emergency Dispatch and Clinical Protocols}

Emergency services worldwide are exploring AI to improve call handling, but these systems must support established clinical triage protocols rather than replace human judgment.

\textbf{Clinical Triage Protocols.} The Emergency Severity Index (ESI) is a five-level acuity scale (Level 1: immediate lifesaving intervention to Level 5: no resources needed) widely used in the United States and internationally \cite{esi_handbook}. Jamaica's Ministry of Health implemented ESI across all 19 public hospital emergency departments in 2016 \cite{french2020}. For mass casualty events such as hurricanes, the START (Simple Triage and Rapid Treatment) protocol provides rapid four-category sorting: BLACK (deceased/expectant), RED (immediate), YELLOW (delayed), and GREEN (walking wounded). The ESI handbook explicitly notes that ESI should not be used during mass casualty incidents \cite{esi_handbook}.

\textbf{Current AI Systems.} Existing emergency AI systems (e.g., ECA \cite{attiah2025}, Corti \cite{blomberg2019}) achieve promising classification accuracy but rely on cloud-dependent, accent-agnostic ASR and process only transcribed text, ignoring paralinguistic signals. A scoping review of 106 AI studies in prehospital care identified underutilization of multimodal inputs and absence of infrastructure-independent systems as key gaps \cite{chee2023}.

\textbf{Gaps for Caribbean Deployment.} Three limitations motivate TRIDENT's design: (1) no accent adaptation for Caribbean varieties or stress-induced register shifting, (2) no integration of vocal stress detection with text classification, and (3) cloud dependency that fails during disasters when emergency services are most needed. TRIDENT addresses these gaps while maintaining the principle that AI should empower dispatchers to apply ESI/START protocols more effectively, not replace clinical judgment.
