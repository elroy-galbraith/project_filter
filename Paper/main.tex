% Main document for Project Filter paper
\documentclass[11pt,a4paper]{article}

% Packages
\usepackage[utf8]{inputenc}
\usepackage[T1]{fontenc}
\usepackage{amsmath,amssymb}
\usepackage{graphicx}
\usepackage{url}
\usepackage{hyperref}
\usepackage{booktabs}
\usepackage{array}
\usepackage{xcolor}
\usepackage{geometry}
\usepackage{fancyvrb}
\usepackage{listings}
\usepackage{authblk}

% Page geometry
\geometry{margin=1in}

% Hyperref setup
\hypersetup{
    colorlinks=true,
    linkcolor=blue,
    citecolor=blue,
    urlcolor=blue,
    breaklinks=true
}

% URL line breaking
\usepackage{xurl}

% Bibliography formatting - helps with overfull boxes
\emergencystretch=1em

% Code listings setup
\lstset{
    basicstyle=\ttfamily\small,
    breaklines=true,
    frame=single,
    backgroundcolor=\color{gray!10}
}

% Document metadata and abstract
% Document metadata - goes in preamble
\title{TRIDENT: A Redundant Architecture for Caribbean-Accented Emergency Speech Triage}

\author{Galbraith, E.}
\author{Sutherland, C.}
\author{Morgan, D.}
\affil{SMG Labs Research Group}

\date{\today}


\begin{document}

\maketitle

% Abstract
\begin{abstract}
    Emergency speech recognition systems exhibit systematic performance degradation on non-standard English varieties, creating a critical gap in services for Caribbean populations. We present TRIDENT (Triage via Dual-stream Emergency Natural language and Tone), a three-layer triage architecture designed for resilience when automatic speech recognition fails. The system combines Caribbean-accent-tuned ASR, local entity extraction via large language models, and bio-acoustic distress detection to route emergency calls based on transcription confidence, semantic content severity, and vocal stress indicators. \textbf{Our key insight is that low ASR confidence, rather than representing system failure, serves as a valuable triage signal---particularly when combined with elevated vocal distress markers indicating a caller in crisis whose speech may have shifted toward basilectal registers.} A complementary insight drives the content severity layer: trained responders and composed bystanders may report life-threatening emergencies without elevated vocal stress, requiring semantic analysis to capture urgency that paralinguistic features miss. We describe the architectural design, theoretical grounding in psycholinguistic research on stress-induced code-switching, and deployment considerations for offline operation during disaster scenarios. \textbf{This paper presents an architectural framework and position paper; empirical validation on Caribbean emergency calls remains future work.} This work establishes a framework for accent-resilient emergency AI that treats dialect variation as a routing feature rather than a transcription bug.
    \end{abstract}

% Keywords
\noindent\textbf{Keywords:} automatic speech recognition, Caribbean English, emergency dispatch, vocal stress detection, creole continuum, edge computing, position paper

\vspace{0.5cm}

% Main sections
\section{Introduction}

When a caller dials emergency services during a crisis, the interaction between human distress and automated systems creates a critical dependency on speech recognition accuracy. Modern automatic speech recognition (ASR) systems, however, exhibit well-documented performance disparities across demographic groups \cite{koenecke2020}. For Caribbean English speakers---a population of over 40 million across the Anglophone Caribbean and diaspora---these disparities compound with a linguistic phenomenon: under acute stress, speakers tend to shift toward basilectal (more creole-heavy) speech registers, precisely the varieties on which ASR systems perform worst.

\subsection{Clinical Context: Established Triage Protocols}

Caribbean health ministries have adopted internationally-validated triage protocols to standardize emergency response. Jamaica's Ministry of Health implemented the Emergency Severity Index (ESI)---a five-level acuity scale widely used in the United States and globally---across all 19 public hospital emergency departments in 2016 \cite{french2020}. The ESI algorithm stratifies patients from Level 1 (immediate lifesaving intervention required) to Level 5 (no resources needed), based on acuity assessment and anticipated resource utilization.

For mass casualty events such as hurricanes, the START (Simple Triage and Rapid Treatment) protocol provides rapid four-category sorting: BLACK (deceased/expectant), RED (immediate), YELLOW (delayed), and GREEN (walking wounded). The ESI handbook explicitly notes that ESI should \emph{not} be used during mass casualty incidents, where START or similar rapid triage systems are appropriate \cite{esi_handbook}.

However, a 2020 evaluation of ESI implementation in Jamaica found poor interrater reliability between Jamaican practitioners and ESI experts, with triage note quality, completeness of vital sign assessment, and high staff attrition identified as key challenges \cite{french2020}. These protocols assume dispatchers and triage nurses can accurately capture caller and patient information---an assumption that fails systematically when ASR systems cannot reliably transcribe Caribbean speech.

\subsection{TRIDENT: Dispatcher-Empowered Architecture}

\textbf{This paper presents TRIDENT (\textbf{T}ranscription and \textbf{R}outing \textbf{I}ntelligence for \textbf{D}ispatcher-\textbf{E}mpowered \textbf{N}ational \textbf{T}riage), an architectural framework} designed not to replace established triage protocols, but to ensure Caribbean-accented callers receive equitable access to them. Rather than attempting to eliminate ASR errors on Caribbean speech---an unrealistic goal given current technology---we build a \textbf{dispatcher-support system} that remains functional when transcription fails. \textbf{We frame this work as a position paper and system proposal}, establishing theoretical foundations and design rationale while acknowledging that end-to-end empirical validation on Caribbean emergency calls remains future work.

\textbf{Our central contribution is a three-layer dispatcher-support framework} that provides human dispatchers with structured inputs for protocol application. The system generates three complementary signals:

\begin{enumerate}
    \item \textbf{Transcription confidence:} Flags unreliable transcripts so dispatchers know to listen directly to call audio rather than relying on text alone.
    
    \item \textbf{Structured entity extraction:} Extracts clinical indicators needed for ESI/START application---location, mechanism of injury, breathing status, number of persons affected, presence of vulnerable populations---even from partial or degraded transcriptions.
    
    \item \textbf{Bio-acoustic distress detection:} Provides a novel signal not currently captured by standard triage protocols---physiological stress markers derived from vocal acoustics that indicate caller crisis state independent of transcript content.
\end{enumerate}

\subsection{Key Insights}

Two complementary insights motivate this design:

\begin{enumerate}
    \item \textbf{Content beyond voice:} Trained first responders, medical professionals, and composed bystanders may report life-threatening emergencies without elevated vocal stress. Semantic extraction of clinical entities captures information that paralinguistic features alone would miss---ensuring that ``children trapped in burning building,'' spoken calmly, provides dispatchers with the structured data needed for appropriate ESI or START classification.
    
    \item \textbf{Uncertainty as prioritization signal:} Low ASR confidence, rather than representing system failure, serves as a valuable indicator for queue prioritization---particularly when combined with elevated vocal distress markers indicating a caller in crisis whose speech may have shifted toward basilectal registers. This reframes accent-induced transcription errors from bugs into features that correlate with genuine caller distress, ensuring these callers receive priority human attention for proper triage assessment.
\end{enumerate}

\subsection{Addressing Gaps in Emergency AI}

The architecture addresses four gaps in existing emergency AI systems:

\begin{enumerate}
    \item \textbf{Cloud dependency:} Existing systems rely on cloud-based, accent-agnostic ASR that fails during infrastructure outages common in disasters.
    
    \item \textbf{Text-only analysis:} Current approaches focus exclusively on textual features, ignoring paralinguistic stress signals that may indicate caller distress even when words are unclear.
    
    \item \textbf{Dialect blindness:} No consideration of dialect continua or stress-induced register shifting that characterizes Caribbean (and other bidialectal) speech communities.
    
    \item \textbf{Infrastructure fragility:} Inability to function during the communication infrastructure failures that commonly accompany the disasters when emergency services are most needed.
\end{enumerate}

Critically, TRIDENT addresses these gaps while \textbf{respecting the clinical authority of established protocols}. The system structures inputs and prioritizes dispatcher queues, but triage decisions remain with trained human professionals applying Ministry of Health-mandated frameworks. TRIDENT empowers dispatchers with better information; it does not replace their judgment.

\subsection{Scope and Generalizability}

\textbf{Note on stress-induced register shift:} While we focus on Caribbean creole continua, the phenomenon of dialect reversion under cognitive load is not unique to this population. The inhibitory control model of bilingual processing \cite{green1998} and research on the Lombard effect (speech modifications in noisy environments) suggest our framework may generalize to other bidialectal populations worldwide. Caribbean emergency services serve as our motivating case study, but the architectural principles apply broadly to any context where:

\begin{itemize}
    \item Speakers use dialectal varieties underrepresented in ASR training data
    \item Stress may induce shifts toward non-standard registers
    \item Established triage protocols require accurate capture of caller information
    \item Infrastructure resilience is critical for emergency response
\end{itemize}
\section{Related Work}

The proposed crisis triage system draws on and extends research across four domains: automatic speech recognition for accented and low-resource speech varieties, artificial intelligence in emergency dispatch, vocal stress detection, and edge computing for disaster resilience. We review each in turn, identifying the gaps that motivate our three-layer architecture.

\subsection{The Accent Gap in Automatic Speech Recognition}

Modern ASR systems exhibit systematic performance degradation on non-standard English varieties---a disparity with serious implications for equitable access to voice-enabled services. Koenecke et al. \cite{koenecke2020} conducted the seminal quantitative study, evaluating five major commercial ASR systems across racial demographics. Their findings were stark: word error rates averaged 0.35 for Black speakers compared to 0.19 for White speakers, with 23\% of Black speaker audio producing WER exceeding 0.50---functionally unusable transcription---compared to just 1.6\% for White speakers. Critically, the researchers traced these disparities to acoustic models rather than language models, as the performance gap persisted even on identical phrases.

The Edinburgh International Accents of English Corpus (EdAcc) benchmark extends this analysis to global accent variation \cite{sanabria2023}. Testing revealed that OpenAI's Whisper-large model achieved 19.7\% WER on EdAcc compared to just 2.7\% on LibriSpeech test-clean---a seven-fold performance degradation on accented speech. The study specifically identified Jamaican English among the accents with highest error rates, directly validating concerns about Caribbean speech recognition.

Research on African-accented English provides methodologically rigorous comparators. The AfriSpeech-200 corpus encompasses 200 hours of Pan-African English speech across 120 accents from 13 Anglophone countries, with evaluations demonstrating that models achieving 1-3\% WER on standard corpora produce 10-90\% WER on African-accented subsets \cite{olatunji2023}. Named entities and domain-specific terminology proved particularly challenging---a finding directly relevant to emergency contexts where accurate location and hazard extraction is critical.

Caribbean English remains especially underserved despite representing millions of speakers. Madden et al. \cite{madden2025} developed the first substantial Jamaican Patois speech corpus (42.58 hours) and derived scaling laws for Whisper performance on this variety. Their results are instructive: pre-trained Whisper Large achieved 89\% WER on Patois---functionally useless---while fine-tuned Whisper Medium reduced this to 30\% WER. Notably, fine-tuned Whisper Tiny outperformed non-fine-tuned Whisper Large, demonstrating that domain-specific data matters more than model size for underrepresented varieties. Their scaling law (WER = 158.06 $\times$ M$^{-0.255}$ $\times$ D$^{-0.269}$) reveals that dataset increases yield greater gains than model scaling for this population, informing our choice of Whisper Medium with Caribbean-specific fine-tuning.

\subsection{AI-Assisted Emergency Dispatch and Clinical Protocols}

Emergency services worldwide are exploring AI to improve call handling, but these systems must support established clinical triage protocols rather than replace human judgment.

\textbf{Clinical Triage Protocols.} The Emergency Severity Index (ESI) is a five-level acuity scale (Level 1: immediate lifesaving intervention to Level 5: no resources needed) widely used in the United States and internationally \cite{esi_handbook}. Jamaica's Ministry of Health implemented ESI across all 19 public hospital emergency departments in 2016 \cite{french2020}. For mass casualty events such as hurricanes, the START (Simple Triage and Rapid Treatment) protocol provides rapid four-category sorting: BLACK (deceased/expectant), RED (immediate), YELLOW (delayed), and GREEN (walking wounded). The ESI handbook explicitly notes that ESI should not be used during mass casualty incidents \cite{esi_handbook}.

\textbf{Current AI Systems.} Existing emergency AI systems (e.g., ECA \cite{attiah2025}, Corti \cite{blomberg2019}) achieve promising classification accuracy but rely on cloud-dependent, accent-agnostic ASR and process only transcribed text, ignoring paralinguistic signals. A scoping review of 106 AI studies in prehospital care identified underutilization of multimodal inputs and absence of infrastructure-independent systems as key gaps \cite{chee2023}.

\textbf{Gaps for Caribbean Deployment.} Three limitations motivate TRIDENT's design: (1) no accent adaptation for Caribbean varieties or stress-induced register shifting, (2) no integration of vocal stress detection with text classification, and (3) cloud dependency that fails during disasters when emergency services are most needed. TRIDENT addresses these gaps while maintaining the principle that AI should empower dispatchers to apply ESI/START protocols more effectively, not replace clinical judgment.

\subsection{Vocal Stress Detection}
\label{sec:vocal_stress}

The bio-acoustic layer builds on research establishing acoustic correlates of psychological stress. A systematic review of 38 studies found fundamental frequency (F0) as the most consistent stress marker, with 15 of 19 studies reporting significant mean F0 increases under stress \cite{schmalz2025}.

Research on emergency communications provides direct validation. Van Puyvelde et al. \cite{vanpuyvelde2018} analyzed real-life emergency recordings including cockpit voice recorders and 911 calls, documenting F0 increases from 123.9 Hz to 200.1 Hz during life-threatening emergencies---a 62\% increase. However, Deschamps-Berger et al. \cite{deschampsberger2021} found that while benchmark IEMOCAP data yielded 63\% emotion recognition accuracy, real emergency calls achieved only 45.6\%---a substantial domain shift. This finding reinforces our design decision to use bio-acoustic analysis as a triage signal routing high-distress calls to human dispatchers, rather than attempting fully automated classification.

\subsection{Dialect Reversion Under Cognitive Load}

A theoretical foundation for Caribbean-specific ASR in emergency contexts comes from psycholinguistic research on bilingual processing under stress. The inhibitory control model establishes that non-target languages remain continuously active and must be suppressed through cognitive effort \cite{green1998}. For Caribbean speakers navigating the creole continuum---from basilect (most creole features) through mesolect to acrolect (Standard English)---maintaining acrolectal speech requires sustained executive function.

\textbf{The creole continuum is not simply a stylistic choice but a dynamic system of linguistic control, modulated by cognitive load.} Research on cognitive load effects demonstrates that this inhibition fails under stress. Gollan and Ferreira \cite{gollan2009} found that under high cognitive load, bilingual speakers use significantly less intraclausal code-switching, instead reverting to monolingual chunks of their dominant language. Importantly, cognitive load also affects lexical access timing---Kroll et al. \cite{kroll2006} demonstrated that retrieval of L2 (non-dominant language) vocabulary slows significantly under dual-task conditions, providing a mechanism for stress-induced register shift.

Patrick's \cite{patrick1999} foundational sociolinguistic analysis of the Jamaican Creole continuum establishes that stress levels influence speakers' positioning on this spectrum, with most speakers being mesolectal in normal conditions but capable of shifting toward either pole.

The implications for emergency services are significant: a professional who speaks Standard English at work may revert toward basilectal Patois when their house is flooding. Standard ASR systems, trained predominantly on acrolectal varieties, will exhibit precisely the performance degradation documented in the accent gap literature at the moment when accurate recognition is most critical. Our system addresses this by fine-tuning on Caribbean broadcast data that includes mesolectal speech, and by providing bio-acoustic fallback when ASR confidence drops---which may itself serve as a proxy indicator for basilectal reversion.

\subsection{Edge Computing for Disaster Resilience}

The case for offline-capable emergency AI is made starkly by infrastructure failure during recent disasters. Hurricane Maria's impact on Puerto Rico saw 95\% of cell towers fail, with the entire island losing power and over 66\% of the population lacking potable water \cite{santosburgoa2020}. Communication infrastructure failure caused delays in mortality reporting and created substantial information vacuums, contributing to a disputed death toll ultimately estimated at approximately 3,000. Recovery required over 200 days for full power restoration.

Recent advances in model compression make edge deployment increasingly feasible. Quantization studies demonstrate that 4-bit (INT4) quantization reduces Whisper model size by 45-87\% with minimal WER degradation, and may actually reduce hallucinations by acting as a regularizer. Gondi and Pratap \cite{gondi2021} demonstrated that transformer-based ASR achieves real-time inference on Raspberry Pi hardware with PyTorch mobile optimization. For the NLP component, 4-bit quantized Llama 3 8B runs at 2-5 tokens per second on Raspberry Pi 5---too slow for real-time conversation but adequate for background entity extraction tasks.

A survey of edge technologies for disaster management identifies prediction, detection, response, and recovery phases where edge computing enables real-time processing without cloud dependency \cite{aboualola2023}. The survey specifically identifies a gap in offline-capable speech and language processing at the edge---precisely the capability our system provides. Pre-positioned edge computing resources at hospitals, shelters, and emergency coordination centers, loaded with Caribbean-tuned models, could maintain triage capability even during complete grid and network failure.

\subsection{Summary: Positioning Our Contribution}

The literature reveals a clear opportunity space for Caribbean emergency services. As introduced in Section~\ref{sec:addressing_gaps}, existing dispatch AI systems exhibit four critical limitations for Caribbean deployment: cloud dependency with accent-agnostic ASR, text-only analysis, dialect blindness, and infrastructure fragility.

\textbf{How the literature establishes each gap:}

\begin{itemize}
    \item \textbf{Cloud dependency:} ECA and Corti rely on commercial cloud APIs (Google Speech-to-Text) with documented performance degradation on non-standard English varieties (Section 2.1, Madden et al. scaling law findings). No existing system adapts for Caribbean accents or creole continua.

    \item \textbf{Text-only analysis:} Current approaches process only transcribed text, ignoring paralinguistic stress signals documented in Section 2.3 (Van Puyvelde et al., Schmalz et al.). This misses critical information when words are unclear or mistranscribed.

    \item \textbf{Dialect blindness:} No existing system accounts for stress-induced register shifting demonstrated in Section 2.4 (inhibitory control model, creole continuum research)---the phenomenon whereby speakers under acute stress revert toward basilectal varieties, precisely when accurate transcription matters most.

    \item \textbf{Infrastructure fragility:} Hurricane Maria case (Section 2.5) demonstrates how cloud-dependent architectures fail during the disasters that generate emergency call surges.
\end{itemize}

\textbf{TRIDENT's contribution} is a dispatcher-support architecture that addresses each gap while respecting the clinical authority of established triage protocols:

\begin{itemize}
    \item \textbf{Caribbean-adapted ASR:} Fine-tuned Whisper models provide the transcription accuracy that makes downstream entity extraction viable for Caribbean speech varieties.
    
    \item \textbf{Structured entity extraction:} Local Llama 3-based NLP extracts clinical indicators needed for ESI/START application---location, mechanism of injury, breathing status, vulnerable populations---operating without internet connectivity.
    
    \item \textbf{Bio-acoustic distress detection:} A parallel signal pathway that functions even when ASR fails, transforming low transcription confidence from a system limitation into a queue prioritization feature that routes distressed callers to immediate human attention.
    
    \item \textbf{Offline operation:} Complete system deployment on edge hardware (Raspberry Pi 5) enables function during infrastructure failures when emergency services are most critical.
\end{itemize}

The result is the first dispatcher-support system designed specifically for Caribbean emergency services---not to make triage decisions, but to ensure that Caribbean-accented callers receive equitable access to the ESI and START protocols that their health ministries have adopted. TRIDENT empowers dispatchers with better information and intelligent queue prioritization; clinical judgment remains where it belongs---with trained human professionals.

\section{System Architecture}

TRIDENT implements a three-layer architecture where each component provides independent value while contributing to a unified triage decision. Figure~\ref{fig:system_diagram} illustrates the system flow.

\begin{figure}[ht]
\centering
\includegraphics[width=0.9\textwidth]{figures/project_filter_architecture.pdf}
\caption{The TRIDENT architecture. The system processes raw audio through two parallel streams: (Left) A Caribbean-adapted ASR and NLP pipeline for semantic extraction and content severity scoring, and (Right) a bio-acoustic analysis layer for determining physiological distress. The Triage Decision Engine integrates three independent signals---ASR confidence, content severity, and vocal distress---ensuring that (1) calls with low transcription confidence but high vocal distress are routed to priority dispatch, and (2) semantically urgent calls from calm reporters are not under-prioritized due to absent vocal stress markers.}
\label{fig:system_diagram}
\end{figure}

\subsection{Layer 1: Caribbean-Tuned ASR}

The ASR layer employs OpenAI's Whisper Large model (769M parameters) fine-tuned with Low-Rank Adaptation (LoRA) on Caribbean broadcast speech. Competition experience suggests that Whisper Large is more accurate than Whisper Medium for Caribbean speech, even though the latter is the default model used by OpenAI. This notwithstanding, for TRIDENT, we selected Whisper Medium over Large based on Madden et al.'s \cite{madden2025} scaling law, which demonstrates diminishing returns from model size compared to domain-specific data for Caribbean varieties. Furthermore, Whisper Medium is more efficient to run on a Raspberry Pi 5, which is the edge device we are using for TRIDENT.

\textbf{Fine-tuning Configuration:}
\begin{itemize}
    \item Base model: openai/whisper-medium
    \item Adaptation: LoRA (rank=16, alpha=32)
    \item Training data: BBC Caribbean broadcast corpus ($\sim$28,000 clips)
    \item Trainable parameters: $\sim$0.5\% of total model
\end{itemize}

\textbf{Confidence Scoring:} The system computes \textbf{utterance-level} confidence as the mean log-probability across all decoded tokens, normalized to a 0-1 scale. Specifically:

\begin{equation}
\text{confidence} = \exp\left(\frac{1}{N}\sum_{i=1}^{N} \log P(t_i | t_1 \ldots t_{i-1}, \text{audio})\right)
\end{equation}

We use utterance-level rather than token-level confidence because emergency triage requires a holistic assessment of transcription reliability. Token-level confidence would require additional aggregation logic and may miss systematic degradation patterns (e.g., consistently low confidence across an entire basilectal utterance).

\textbf{Confidence Threshold:} We set the ``low confidence'' threshold at 0.7 based on initial calibration experiments, though sensitivity analysis is needed to optimize this value (see Limitations).

The configuration above reflects design specifications informed by the Caribbean Voices AI Hackathon, which provided access to the BBC Caribbean corpus. Competition data rights are retained by the organizers; empirical validation therefore remains future work. The architecture accommodates any Caribbean-tuned Whisper model meeting these specifications.

\subsection{Layer 2: Local NLP Entity Extraction}

When ASR produces usable transcription (confidence $\geq$ 0.7), the NLP layer extracts structured emergency information using Llama 3 8B running locally via Ollama. The extraction schema targets entity types that map directly to ESI and START triage protocol decision points.

\subsubsection{Entity Extraction Schema}

The schema targets four entity categories:
\begin{itemize}
    \item \textbf{LOCATION:} Street addresses, landmarks, geographic references
    \item \textbf{MECHANISM/HAZARD:} Emergency type (fire, flood, medical, violence, traffic)
    \item \textbf{CLINICAL INDICATORS:} Breathing status, consciousness, bleeding, mobility
    \item \textbf{SCALE:} Number of people involved, vulnerable populations
\end{itemize}

\subsubsection{Mapping to Triage Protocols}

TRIDENT entities support ESI and START protocol application. For ESI, extracted entities inform the four decision points: Point A (lifesaving intervention) captures "not breathing," "choking," "unresponsive"; Point B (high-risk situation) captures mechanism of injury and altered status; Point C (resource needs) uses hazard type and complexity; Point D (vital signs) uses reported vitals and distress indicators \cite{esi_handbook}.

For mass casualty events using START, entities support rapid sorting: GREEN captures "walking," "minor injuries"; YELLOW captures "injured but stable," "conscious"; RED captures "trapped," "not breathing," "heavy bleeding"; BLACK captures cessation indicators.

\begin{table}[ht]
\centering
\small
\begin{tabular}{@{}p{2.5cm}p{4cm}p{5.5cm}@{}}
\toprule
\textbf{Protocol} & \textbf{Decision Point} & \textbf{Example Extraction Target} \\ \midrule
ESI Level 1 & Immediate lifesaving intervention? & ``not breathing,'' ``choking,'' ``heavy bleeding,'' ``unresponsive'' \\[0.5em]
START RED & Not walking, breathing issues & ``trapped,'' ``not breathing,'' ``unresponsive,'' ``heavy bleeding'' \\ \bottomrule
\end{tabular}
\caption{Example entity extraction targets supporting ESI and START protocols. Full protocol mappings detailed in extended version.}
\label{tab:protocol_mapping_simplified}
\end{table}

\subsubsection{Handling Garbled Input}

The NLP layer handles low-quality transcriptions through confidence-aware prompting. When ASR confidence is below 0.7, the system instructs the LLM to mark uncertain extractions, avoid hallucination, prioritize location extraction, and note phonetically similar alternatives. When confidence is very low ($<$0.4), minimal structured output is produced and the call is flagged for immediate human review.

\subsubsection{Content Indicator Scoring}

The NLP layer computes a \textbf{Content Indicator Score} ($S_c \in [0,100]$) quantifying urgency implied by semantic content, independent of how the caller sounds. This addresses a critical gap: a trained first responder may report a mass casualty event calmly, producing low bio-acoustic distress despite extremely urgent content. Without content analysis, such calls would be deprioritized.

Rather than keyword matching, we leverage the LLM's semantic understanding to classify transcript content. This approach handles Caribbean creole variants (``mi granmodda drop dung an she nah move'' conveys the same urgency as ``my grandmother collapsed and she's not moving''), negation, and indirect references.

The LLM outputs structured classifications:
\begin{verbatim}
{
  "hazard_category": "violent_crime" | "medical" | "fire" |
                     "flood" | "traffic" | "infrastructure" | "other",
  "life_threat_level": "imminent" | "potential" | "none",
  "vulnerable_population": true | false,
  "situation_status": "escalating" | "stable" | "resolved",
  "persons_affected": <integer>
}
\end{verbatim}

A deterministic function maps classifications to the score:
\begin{equation}
S_c = \min\left(100,\ S_{\text{hazard}} + S_{\text{threat}} + S_{\text{vuln}} + S_{\text{scale}}\right)
\label{eq:content_severity}
\end{equation}

\textbf{Scoring components:} Hazard category weights range from 30 (violent crime) to 5 (other). Life-threat level contributes +30 (imminent), +15 (potential), or +0 (none). Vulnerable population adds +15. Scale combines persons affected (+5 per person, capped at +20) and escalation status (+10 if escalating).

\textbf{Example calculations:}

\begin{table}[ht]
\centering
\small
\begin{tabular}{@{}p{0.40\textwidth}p{0.35\textwidth}c@{}}
\toprule
\textbf{Transcript} & \textbf{Classification} & \textbf{$S_c$} \\ \midrule
``Pothole on Nelson Street'' &
infrastructure, none, false, stable, 0 & 10 \\[0.5em]
``House fire, spreading to neighbor's yard'' &
fire, potential, false, escalating, 0 & 50 \\[0.5em]
``Pickney dem trap inna di fire'' &
fire, imminent, true, stable, 2+ & 80 \\ \bottomrule
\end{tabular}
\caption{Content indicator scoring via LLM classification. Semantic understanding captures urgency from Caribbean creole variants. High scores elevate queue priority; clinical triage remains with dispatchers.}
\label{tab:severity_examples}
\end{table}

The Content Indicator Score feeds into queue prioritization (Section~\ref{sec:queue_prioritization}), ensuring semantically urgent calls reach dispatchers promptly even when vocal distress markers are absent. Weights are tunable parameters that should be calibrated with local emergency services to reflect institutional priorities and regional hazard profiles.

\subsection{Layer 3: Bio-Acoustic Distress Detection}

The bio-acoustic layer operates on raw audio, independent of ASR success, extracting features correlated with psychological distress. Based on the vocal stress literature \cite{schmalz2025, vanpuyvelde2018, veiga2025}, we focus on features that capture physiological arousal through vocal production changes.

\subsubsection{Feature Extraction}

Using librosa, we extract the following acoustic features:

\begin{enumerate}
    \item \textbf{Fundamental Frequency (F0):} Mean pitch extracted via autocorrelation method
    \begin{itemize}
        \item Typical baseline: 85--180 Hz (male), 165--255 Hz (female) \cite{titze1989}
        \item Stress indicator: Elevation above speaker baseline
    \end{itemize}

    \item \textbf{F0 Coefficient of Variation (CV):} Pitch instability measure
    \begin{itemize}
        \item Computed as $CV = \sigma_{F0} / \mu_{F0}$
        \item Normalizes for baseline differences across speakers
        \item Stress indicator: $CV > 0.3$ suggests vocal instability
    \end{itemize}

    \item \textbf{Energy (RMS amplitude):} Mean intensity across utterance
    \begin{itemize}
        \item Normalized to 0--1 scale relative to recording gain
        \item Stress indicator: Elevated intensity during distress vocalizations
    \end{itemize}

    \item \textbf{Jitter:} Cycle-to-cycle variation in F0 period
    \begin{itemize}
        \item Relatively independent of prosodic patterns \cite{vanpuyvelde2018}
        \item Pathology threshold: $>$1.04\% \cite{boersma2013}
    \end{itemize}
\end{enumerate}

\subsubsection{Distress Score Calculation}

The distress score combines multiple acoustic indicators into a composite metric. We weight features according to their documented reliability and sex-independence:

\begin{align}
D &= w_{\text{pitch}} \cdot P + w_{\text{var}} \cdot V + w_{\text{energy}} \cdot E + w_{\text{jitter}} \cdot J
\label{eq:distress}
\end{align}

\noindent where:

\begin{align}
P &= \min\left(1.0, \max\left(0, \frac{\bar{F_0} - 180}{120}\right)\right) & \text{(pitch elevation)} \\
V &= \min\left(1.0, \frac{CV_{F0}}{0.5}\right) & \text{(pitch instability)} \\
E &= \min\left(1.0, \frac{\bar{E}}{0.1}\right) & \text{(energy)} \\
J &= \min\left(1.0, \frac{\text{jitter}}{0.02}\right) & \text{(perturbation)}
\end{align}

The weights reflect relative reliability from the literature:
\begin{itemize}
    \item $w_{\text{pitch}} = 0.30$ --- F0 elevation is the most consistent stress marker but is sex-dependent
    \item $w_{\text{var}} = 0.35$ --- F0 coefficient of variation is sex-normalized and robust
    \item $w_{\text{energy}} = 0.20$ --- intensity elevation accompanies distress
    \item $w_{\text{jitter}} = 0.15$ --- perturbation measures are prosody-independent
\end{itemize}

\subsubsection{Threshold Classification}

\begin{itemize}
    \item \textbf{High Distress:} $D > 0.5$
    \item \textbf{Low Distress:} $D \leq 0.5$
\end{itemize}

These thresholds are calibrated against Van Puyvelde et al.'s \cite{vanpuyvelde2018} findings on vocal markers in emergency versus baseline speech.

\textbf{Note on sex differences:} The distress score prioritizes sex-normalized features (CV, jitter) over absolute F0 elevation to mitigate the substantial baseline differences between male (85--175 Hz) and female (165--270 Hz) speakers. See Section~\ref{sec:sex_limitations} for detailed discussion of remaining bias risks.
\subsection{The Complementarity Principle}

The theoretical foundation for our multi-layer design rests on what we term the \textbf{Complementarity Principle}: the three signal dimensions capture distinct failure modes and urgency indicators that compensate for each other's blind spots, ensuring dispatchers receive the most critical calls first regardless of which individual signal might fail.

\textbf{Dimension 1: Transcription Confidence.} The conditions that degrade ASR performance (high stress, code-switching to basilect, environmental noise) are precisely the conditions that often accompany genuine emergencies. Low confidence is not merely a technical limitation to be hidden---it correlates with caller distress and should elevate queue priority while flagging the call for direct audio review.

\textbf{Dimension 2: Content Indicators.} Semantic analysis of transcript content captures urgency that vocal characteristics may miss. Trained professionals, repeat callers, and composed bystanders often report critical emergencies without elevated vocal stress---their calm delivery masks the urgency that only content analysis reveals. When transcription confidence is high, extracted entities map directly to ESI/START decision points.

\textbf{Dimension 3: Bio-Acoustic Distress.} Vocal stress markers (elevated pitch, intensity, instability) provide a parallel assessment channel that operates on raw audio, independent of transcription success. A caller whose speech is entirely unintelligible to ASR will still produce detectable distress signals. This dimension captures information not currently used by ESI or START protocols, representing TRIDENT's novel contribution to dispatcher awareness.

This creates a robust prioritization space with complementary coverage:

\textbf{Dimensional ordering.} The three dimensions are evaluated in deliberate sequence: \emph{Confidence}, \emph{Content}, \emph{Concern}. This ordering reflects operational logic: (1) \emph{Can we understand the caller?}---ASR confidence determines whether transcription is reliable enough for downstream analysis; (2) \emph{What is being reported?}---semantic content establishes the substance of the emergency; (3) \emph{How distressed does the caller sound?}---bio-acoustic indicators validate and can elevate priority, but do not override content. This sequence ensures that a composed professional reporting a mass casualty event receives appropriate priority based on content, while a highly distressed caller reporting a minor issue is not over-prioritized based on vocal expression alone.

\begin{itemize}
    \item \textbf{High Confidence + Low Content + Low Concern:} Routine call; dispatcher applies ESI using extracted entities at normal pace
    
    \item \textbf{High Confidence + High Content + Low Concern:} The composed reporter---urgent content from a calm caller requires elevated queue position; dispatcher reviews entities and applies ESI, likely assigning ESI-2 or ESI-3
    
    \item \textbf{High Confidence + Low Content + High Concern:} Anxious caller, possibly minor issue---dispatcher assesses whether distress reflects emergency or anxiety
    
    \item \textbf{High Confidence + High Content + High Concern:} All signals aligned; immediate queue position for rapid ESI/START application
    
    \item \textbf{Low Confidence + Low Content + Low Concern:} Likely technical issue; dispatcher reviews audio quality before processing
    
    \item \textbf{Low Confidence + High Content + Low Concern:} Garbled but fragments suggest urgency---elevated priority; dispatcher listens directly
    
    \item \textbf{Low Confidence + Low Content + High Concern:} Distressed caller with unintelligible speech---immediate priority; dispatcher listens and applies protocol based on direct assessment
    
    \item \textbf{Low Confidence + High Content + High Concern:} Maximum queue priority---all indicators suggest crisis; immediate dispatcher attention
\end{itemize}

Two cells represent our key insights. The \textbf{High Confidence + High Content + Low Concern} cell captures callers whose semantic content demands urgent attention despite calm delivery: the trained first responder, medical professional, or composed bystander whose measured voice belies the severity of their report. The \textbf{Low Confidence + Low Content + High Concern} cases capture the complementary pattern---callers in crisis whose speech has shifted toward basilectal registers, where ASR failure combined with vocal stress becomes valuable prioritization information rather than system failure.

Together, these insights ensure that neither semantic nor paralinguistic signals alone determine queue position---and that clinical triage decisions remain with trained dispatchers who can assess the full context of each call.

\subsection{Triage Decision Matrix}

The final routing decision combines ASR confidence and distress score according to the following matrix:

\begin{table}[ht]
\centering
\begin{tabular}{@{}lll@{}}
\toprule
\textbf{ASR Confidence} & \textbf{Distress Score} & \textbf{Triage Decision} \\ \midrule
High ($\geq$0.7) & Low ($\leq$0.4) & \textbf{STANDARD:} Queue with extracted metadata \\
High ($\geq$0.7) & Moderate (0.4-0.7) & \textbf{ELEVATED:} Priority queue, human review recommended \\
High ($\geq$0.7) & High ($>$0.7) & \textbf{URGENT:} Immediate human dispatch \\
Low ($<$0.7) & Low ($\leq$0.4) & \textbf{UNCLEAR:} Re-prompt caller, check audio quality \\
Low ($<$0.7) & Moderate (0.4-0.7) & \textbf{PRIORITY REVIEW:} Human dispatcher reviews audio \\
Low ($<$0.7) & High ($>$0.7) & \textbf{CRITICAL:} Immediate human dispatch, flag as potential dialect reversion \\ \bottomrule
\end{tabular}
\caption{Triage decision matrix based on ASR confidence and bio-acoustic distress}
\label{tab:triage_matrix}
\end{table}

\subsubsection{Dispatcher Interface Examples}

Figure~\ref{fig:ui_low_risk} and Figure~\ref{fig:ui_high_risk} illustrate the dispatcher interface for contrasting triage scenarios. The interface presents real-time triage indicators, extracted location metadata, and confidence scores to support rapid human decision-making.

\begin{figure}[ht]
\centering
\includegraphics[width=0.85\textwidth]{figures/Screenshot 2025-11-30 at 8.41.20.png}
\caption{Dispatcher interface for a low-risk scenario (STANDARD triage). The system displays high ASR confidence with successfully extracted location metadata and low distress indicators, allowing the call to queue normally with automated metadata available to the dispatcher.}
\label{fig:ui_low_risk}
\end{figure}

\begin{figure}[ht]
\centering
\includegraphics[width=0.85\textwidth]{figures/Screenshot 2025-11-30 at 8.42.43.png}
\caption{Dispatcher interface for a high-risk scenario (CRITICAL or URGENT triage). Elevated distress markers and reduced ASR confidence trigger immediate priority routing, with visual indicators alerting dispatchers to potential dialect shift or acute crisis requiring immediate human attention.}
\label{fig:ui_high_risk}
\end{figure}

\section{Theoretical Foundations}

\subsection{Why Accent-Tuned ASR Is Necessary But Insufficient}

Fine-tuning Whisper on Caribbean speech will improve transcription accuracy, but it cannot eliminate the accent gap entirely. Madden et al. \cite{madden2025} achieved 30\% WER on Jamaican Patois with fine-tuning---a dramatic improvement from 89\% baseline, but still far above the $<$5\% WER typical for standard English. In emergency contexts, even 30\% WER means nearly one-third of words may be incorrect, potentially including critical location or hazard information.

Moreover, fine-tuning on broadcast speech cannot fully capture emergency speech characteristics: elevated noise (sirens, screaming, wind), emotional vocal qualities, and the stress-induced basilectal reversion discussed above. A system relying solely on ASR, no matter how well-tuned, will fail precisely when it is needed most.

\subsection{Why Bio-Acoustic Analysis Is Necessary But Insufficient}

Conversely, bio-acoustic distress detection alone cannot provide the semantic information needed for emergency dispatch. A caller may exhibit extreme vocal stress while saying ``my house is on fire'' or ``I lost my keys''---the distress signal is identical, but the appropriate response differs dramatically.

Furthermore, as Deschamps-Berger et al. \cite{deschampsberger2021} demonstrated, laboratory accuracy of emotion recognition systems (63\%) drops substantially in real emergency calls (45.6\%). Bio-acoustic features provide reliable \emph{gradient} information about caller state but cannot substitute for semantic content.

\subsection{The Integration Thesis}

Our architecture integrates these complementary information sources based on the following thesis: \textbf{In emergency contexts, the correlation between ASR failure and genuine distress creates an opportunity to use recognition uncertainty as a routing signal rather than an error to be minimized.}

This thesis rests on the psycholinguistic literature establishing that:
\begin{enumerate}
    \item Stress triggers cognitive load effects that impair executive function \cite{gollan2009}
    \item Impaired executive function leads to reduced inhibition of dominant language varieties \cite{green1998}
    \item For Caribbean speakers, dominant varieties include basilectal forms underrepresented in ASR training \cite{patrick1999, madden2025}
    \item Stress simultaneously elevates bio-acoustic markers (F0, intensity) that can be detected independently of speech content \cite{vanpuyvelde2018}
\end{enumerate}

The logical conclusion: when ASR confidence drops and bio-acoustic distress rises, the system has detected a caller in genuine crisis whose speech has shifted beyond standard recognition capabilities. This combination should trigger immediate human review---not because the system has failed, but because it has successfully identified a caller who needs human attention most.

\section{Deployment Considerations}

\subsection{Operational Context: Supporting Protocol Application}

TRIDENT integrates with existing emergency dispatch workflows to support standardized triage protocols---ESI for routine operations, START for mass casualty incidents. \textbf{Day-to-day (ESI context):} TRIDENT processes incoming calls to extract structured entities (location, mechanism, clinical indicators) and assigns queue priority. Dispatchers apply ESI to determine clinical acuity level (1--5) and appropriate response. \textbf{Mass casualty events (START context):} During hurricanes or earthquakes, TRIDENT's queue prioritization manages call surges when volume exceeds dispatcher capacity, enabling rapid caller sorting even when transcription quality degrades. \textbf{Key principle:} TRIDENT determines which calls dispatchers see first and what structured information they receive; clinical triage decisions remain with trained professionals applying Ministry of Health protocols.

\subsection{Primary Deployment: Surge Queue Prioritization}

TRIDENT's greatest value emerges during \textbf{disaster surge conditions}---hurricanes, earthquakes, floods---when call volume exceeds dispatcher capacity and callers must wait in queue. TRIDENT's processing latency (45--60 seconds on edge hardware) precludes real-time transcription, but surge queues provide ideal operational context.

\textbf{Operational flow:}
\begin{enumerate}
    \item Caller dials emergency services; all dispatchers engaged
    \item Caller enters queue and hears automated message requesting description
    \item Caller provides initial statement (15--30 seconds)
    \item TRIDENT processes audio while caller waits (45--60 seconds)
    \item Queue reordered by priority (Q1-IMMEDIATE through Q5-ROUTINE)
    \item Highest-priority call routes first when dispatcher becomes available
    \item Dispatcher receives transcription, extracted entities, and distress indicators to support ESI/START application
\end{enumerate}

\textbf{Why this context maximizes value:} Calls are waiting regardless---TRIDENT uses wait time productively. Queue prioritization ensures most critical callers reach dispatchers first. Extracted entities enable faster protocol application. Low ASR confidence flags alert dispatchers to potential dialect shift or audio quality issues before engagement.

This deployment model represents TRIDENT's primary design target. Caribbean emergency services face predictable annual surge events (hurricane season, June--November) where this capability would directly impact response effectiveness.

\subsection{Early Exit for Critical Cases}

To provide faster routing for clearly distressed callers, the system implements early exit when:
\begin{enumerate}
    \item \textbf{High Distress + Low Confidence:} If $D > 0.8$ and $C < 0.4$, route immediately to Q1-IMMEDIATE. This captures callers exhibiting extreme vocal stress whose speech has likely shifted to basilectal registers.

    \item \textbf{Extreme Distress:} If $D > 0.9$ regardless of confidence, route to Q1-IMMEDIATE.
\end{enumerate}

Under early exit, ASR and bio-acoustics complete in approximately 12 seconds (with bio-acoustic extraction parallel to transcription), reducing Time-to-Q1 from 55 seconds to 12 seconds for clearly distressed callers---a critical improvement for surge queue scenarios.

\subsection{Offline Operation}

All components operate without internet connectivity: Whisper model weights and Llama 3 stored locally, bio-acoustic analysis uses standard signal processing libraries, and queue logic implemented in local Python. This enables deployment at emergency coordination centers that may lose connectivity during disasters while maintaining local power (generator/battery backup). Offline capability ensures TRIDENT can support ESI/START protocol application precisely when infrastructure degradation makes accurate call processing most difficult.

\subsection{Integration with Existing Dispatch Systems}

TRIDENT operates as a \textbf{pre-processing layer} integrating with existing Computer-Aided Dispatch (CAD) systems. The system accepts audio streams, processes them through the three-layer architecture, and outputs structured data packages (queue priority, transcription with confidence, extracted entities, distress indicators) to CAD systems. Dispatchers receive calls in priority order and apply ESI or START protocols using TRIDENT's structured data and/or direct audio review. This requires no changes to clinical protocols---only familiarization with TRIDENT's output format.

\subsection{Hardware Requirements}

The complete system deploys on Raspberry Pi 5 (8GB RAM) or equivalent edge hardware:

\begin{table}[ht]
\centering
\begin{tabular}{@{}llll@{}}
\toprule
\textbf{Component} & \textbf{Model} & \textbf{Size} & \textbf{Inference Speed} \\ \midrule
ASR & Whisper Medium (INT4) & $\sim$400MB & $\sim$10s per 30s audio \\
NLP & Llama 3 8B (4-bit) & $\sim$4GB & 2-5 tokens/sec \\
Bio-acoustic & librosa + numpy & $<$50MB & Real-time \\ \bottomrule
\end{tabular}
\caption{Hardware requirements for edge deployment}
\label{tab:hardware}
\end{table}

Total system footprint: $\sim$4.5GB, well within Raspberry Pi 5 8GB capacity.

\section{Limitations and Future Work}

\subsection{Current Limitations}

\textbf{Validation gap (most critical).} This paper presents an architectural framework with theoretical grounding but limited empirical validation on real emergency calls. Performance claims for each layer are based on component evaluations and related literature rather than end-to-end system testing. The three-dimensional queue prioritization matrix (ASR confidence $\times$ distress $\times$ content indicators) is theoretically motivated but has not been validated against expert dispatcher judgments.

\textbf{Protocol integration.} While this paper frames TRIDENT as a dispatcher-support system for ESI and START protocol application, the entity extraction schema and queue prioritization logic were developed independently of clinical stakeholder input. Full integration with Ministry of Health workflows would require:
\begin{itemize}
    \item Validation that extracted entities map correctly to ESI decision points A--D
    \item Confirmation that queue priority levels align with operational dispatcher workflows
    \item Assessment of whether bio-acoustic distress indicators provide actionable information beyond what dispatchers already perceive
    \item Training material development for dispatcher familiarization with TRIDENT outputs
\end{itemize}

\noindent This clinical integration work represents essential future collaboration with Caribbean emergency services professionals. The current paper establishes technical feasibility; operational validation requires partnership with the health ministries whose protocols TRIDENT aims to support.

\textbf{Training data constraints.} Caribbean emergency speech corpora do not exist. ASR fine-tuning was performed on broadcast speech, which differs significantly from emergency call acoustics in noise profiles, emotional content, and register distribution. The gap between training domain (broadcast) and deployment domain (emergency calls) may introduce systematic errors not captured in current evaluation.

\textbf{Bio-acoustic threshold calibration.} Distress detection thresholds are derived from literature on non-Caribbean, predominantly Western populations. Baseline vocal characteristics may vary across Caribbean demographics, requiring population-specific calibration.

\subsection{Sex Differences in F0 Baseline}
\label{sec:sex_limitations}

Fundamental frequency is sexually dimorphic: male voices typically range 85--175 Hz while female voices range 165--270 Hz \cite{titze1989, traunmuller1995}. Setting a single absolute F0 threshold for distress detection risks differential sensitivity across speaker sex.

\textbf{Architectural mitigation strategies:}
\begin{itemize}
    \item Prioritizing sex-normalized features: F0 coefficient of variation ($CV = \sigma_{F0} / \mu_{F0}$) captures pitch instability independent of baseline; jitter measures cycle-to-cycle perturbations that are ``relatively independent from prosodic patterns'' \cite{vanpuyvelde2018}
    \item Weighting normalized features (CV: 0.35, jitter: 0.15) more heavily than absolute F0 elevation (0.30) in the distress score calculation
\end{itemize}

Research confirms that stress manifests with ``striking parallels in men and women'' \cite{pisanski2018}---both sexes show increased pitch mean, minimum, and variation under acute stress. The challenge is not that stress manifests differently, but that baseline values differ.

\textbf{Residual bias risks:}
\begin{itemize}
    \item \textbf{False positive risk:} A relaxed female speaker near the upper baseline range may contribute to elevated distress scores
    \item \textbf{False negative risk:} A stressed male speaker with naturally low F0 may not contribute sufficiently to the pitch component
\end{itemize}

Automatic sex identification from voice is itself an imperfect classifier, particularly for voices near the overlap region of male and female F0 distributions. Rather than introduce a potentially error-prone sex classification step, we employ the sex-normalized feature strategy above. A validation study with sex-stratified analysis on Caribbean emergency calls is essential to: (1) calibrate population-appropriate thresholds, (2) confirm that normalized measures maintain sensitivity across speaker demographics, and (3) determine whether Caribbean populations exhibit different baseline distributions requiring adjustment.

\textbf{Content indicator classification.} The Content Indicator Score depends on LLM classification quality. While leveraging Llama 3's semantic understanding avoids brittle keyword matching, it introduces new failure modes:
\begin{itemize}
    \item Classification errors propagate deterministically to queue priority
    \item Caribbean creole expressions not well-represented in LLM training data may be misclassified
    \item The model may fail to recognize culturally-specific threat indicators or landmarks
\end{itemize}

\noindent Empirical evaluation of classification accuracy on Caribbean emergency transcripts is needed, with particular attention to false negatives (urgent content classified as non-urgent) that could delay dispatcher attention to critical calls.

\textbf{Single-speaker assumption.} The current architecture assumes single-speaker input. Multi-party calls, common in emergencies (``put your mother on the phone''), are not handled. Speaker changes mid-call could confuse bio-acoustic analysis and entity extraction continuity.

\textbf{Threshold sensitivity.} Multiple thresholds govern system behavior: ASR confidence (0.7), distress score (0.5), and content indicators (50). These values were selected based on literature and initial calibration but have not been rigorously optimized. Sensitivity analysis examining system performance across threshold combinations is needed to understand precision-recall tradeoffs for each queue priority level.

\subsection{Future Work}

\textbf{Clinical stakeholder collaboration.} The most important next step is partnership with Caribbean emergency services to validate TRIDENT's utility in real dispatch workflows. This includes:
\begin{itemize}
    \item Observation studies of current ESI/START application challenges
    \item Dispatcher feedback on extracted entity usefulness and queue priority alignment
    \item Iterative refinement of the entity extraction schema based on clinical input
    \item Development of dispatcher training materials for TRIDENT integration
\end{itemize}

\textbf{Caribbean Emergency Speech Corpus.} A critical enabler for future progress is a dedicated corpus combining Caribbean-accented speech with emergency domain content and stress annotations. Multiple approaches could address this data scarcity, each with distinct tradeoffs. Partnerships with Caribbean emergency services to record and annotate real emergency calls would provide the most ecologically valid data but face substantial privacy, consent, and ethical barriers. Acted emergency scenarios avoid these concerns but may not capture authentic stress-induced vocal characteristics or naturalistic basilectal shift patterns.

A gamified speech elicitation platform represents one potential middle ground. We are exploring \textit{VoicefallJA}, a prototype designed to collect stressed Caribbean speech through game-induced cognitive load rather than acted performance.

\textbf{Game Design.} VoicefallJA presents falling-word targets that players must speak aloud before words exit the screen. Difficulty progression (increased speed, shorter response windows) induces naturalistic cognitive load, eliciting stress responses without deception. Phrase prompts span the creole continuum---from acrolectal (``The hospital is on Nelson Street'') through mesolectal (``Di hospital deh pon Nelson Street'') to basilectal (``Di haspital deh dung a Nelson Street side'')---enabling register-annotated collection.

\textbf{Distribution.} The Progressive Web App is designed for WhatsApp-based distribution through institutional partnerships, specifically Methodist Church in the Caribbean and the Americas (MCCA) congregational networks spanning Jamaica and Montserrat. Target: 100--300 speakers, 5,000+ utterances.

\textbf{Ethical Framework.} Our consent model includes: (1) persistent recording indicators, (2) ``panic button'' mid-session revocation, (3) play-only mode without audio recording, and (4) community benefit mechanisms ensuring Caribbean communities receive value from their linguistic contributions.

\textbf{Annotation Schema.} Each utterance is annotated with: register label (ACR/MES/BAS), game-induced stress level (1--5), prosodic features (F0 mean/SD, jitter, shimmer, speech rate), and optional demographics.

\textbf{Timeline.} Q1 2026: beta launch; Q2--Q3 2026: data collection; Q3 2026: dataset release targeting 20+ hours of register-annotated speech.

\textbf{Limitations of the gamification approach.} While VoicefallJA addresses some data collection barriers, game-induced stress differs fundamentally from genuine emergency distress. The platform cannot replicate the life-threatening context, physiological arousal patterns, or communicative urgency of actual emergency calls. Additionally, self-selected participants comfortable with technology may not represent the full demographic diversity of emergency callers. This approach should be viewed as a stepping stone toward more ecologically valid data collection methods, not a replacement for partnerships with emergency services that could enable real-call annotation under appropriate ethical frameworks.

\textbf{Empirical validation.} End-to-end evaluation with emergency dispatch professionals assessing whether TRIDENT's queue prioritization aligns with expert judgment. This should include:
\begin{itemize}
    \item Comparison of three-dimensional prioritization against two-dimensional (confidence $\times$ distress) baseline
    \item Sex-stratified analysis of bio-acoustic distress detection accuracy
    \item Assessment of entity extraction accuracy on Caribbean creole transcripts
    \item Measurement of dispatcher efficiency gains (if any) when using TRIDENT outputs
\end{itemize}

\textbf{Ablation studies.} Rigorous testing to quantify the contribution of each architectural component:
\begin{itemize}
    \item Does bio-acoustic analysis improve queue prioritization over ASR-only approaches?
    \item Do content indicators catch urgent calls missed by distress detection alone?
    \item What is the marginal value of Caribbean-tuned ASR versus off-the-shelf Whisper?
\end{itemize}

\textbf{Sex-adaptive distress detection.} Implementing and validating approaches to further reduce sex bias:
\begin{itemize}
    \item Within-call F0 \textit{change} detection rather than absolute thresholds
    \item Automatic speaker characteristic estimation for threshold adaptation
    \item Ensemble approaches combining multiple normalization strategies
\end{itemize}

\textbf{Dialect density estimation.} Augmenting the system with automatic estimation of creole feature density, providing dispatchers with guidance on expected communication challenges and informing decisions about when to rely on extracted text versus direct audio review.

\textbf{Multilingual extension.} Caribbean emergency services handle calls in English, Spanish, French, Dutch, and various creoles. Extending the architecture to multilingual operation would significantly expand impact, though each language introduces its own ASR adaptation and entity extraction challenges.

\textbf{Edge deployment optimization.} While the architecture is designed for offline operation, current latency profiles (45--60 seconds per call) limit real-time applicability. Optimization for edge hardware (Raspberry Pi, embedded GPU) would enable faster queue prioritization at emergency coordination centers operating with degraded connectivity.
\section{Conclusion}

TRIDENT presents a defensive architecture for Caribbean emergency speech processing that treats ASR limitations not as failures to be eliminated but as signals to be incorporated into triage logic. By combining accent-adapted speech recognition, local NLP extraction, and bio-acoustic distress detection, the system maintains functionality across a range of conditions---including the high-stress, dialect-shifted speech most likely to defeat traditional ASR approaches.

The key insight is that low ASR confidence combined with high vocal distress is not a system failure but a system feature: the signature of a caller in genuine crisis whose speech patterns have shifted beyond the reach of standard recognition. Routing these calls to priority human review ensures that the most vulnerable callers receive the most urgent attention.

We hope this architectural framework contributes to more equitable emergency AI systems---not just for Caribbean populations, but for the billions of speakers worldwide whose accents and dialects remain underserved by current speech technology.


% References
\bibliographystyle{plain}
\bibliography{refs/references}

% Appendices
\appendix
\section{Implementation Details}

\textbf{Repository:} \url{https://github.com/smg-labs/project-filter} \emph{(to be made public upon acceptance)}

\textbf{Dependencies:}
\begin{itemize}
    \item Python 3.11+
    \item openai-whisper
    \item transformers, peft (LoRA fine-tuning)
    \item ollama (Llama 3 serving)
    \item librosa (audio feature extraction)
    \item jiwer (WER evaluation)
\end{itemize}

\textbf{Hardware requirements:}
\begin{itemize}
    \item Training: NVIDIA GPU with 16GB+ VRAM recommended
    \item Inference: CPU-only operation supported; 8GB RAM minimum
\end{itemize}

\section{Acknowledgments}

This work emerged from the Caribbean Voices AI Hackathon, organized by the UWI AI Innovation Centre and hosted on Zindi as part of the Caribbean Information and Youth Network (CIYN), a strategic initiative of the Caribbean Telecommunications Union (CTU). We thank the sponsors—CIBC, Infolytics, and DataAxis—for their support. The competition's BBC Caribbean speech corpus (28,000 clips) motivated this architectural framework, though we note that TRIDENT is presented as a position paper; the organizers retain rights to fine-tuned models from the competition. We also thank Dr. Sikopo Nyambe-Galbraith for feedback on the research.


\end{document}
